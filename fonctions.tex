%%%%%%%%%%%%%%%%%%%%%%%%%%%%%%%%%%%%%%%%%%%%%%%%%%%%%%%%%%%%%%%%%%%%%%%%%%%%%%%%%%%%%%%%%%%%%%
\begin{comment}
	Fonction utilisée pour insérer une image. L'image doit être située dans le répertoire "img"
	Paramètres : 
	#1 : Positionnement de l'image [Paramètre optionnel]
			H 	Place le flottant ici, c'est-à-dire à l'endroit auquel il apparaît dans le texte source.
			t 	Position en haut de la page.
			b 	Position en bas de la page.
			p 	Place sur une page particulière réservée aux flottants.
			! 	Passe outre les paramètres internes que Latex utilise pour déterminer une position optimale des flottants. (déconseillé)
	#2 : échelle de l'image
	#3 : nom absolu du fichier (extension du fichier comprise)
	#4 : Commentaire affiché sous l'image
	#5 : id de l'image (utile pour y faire référence par la suite)
	Exemple d'utilisation :
	\newImage{2}{test.png}{ceci est un test}{test} : Laisser LateX se demerder pour placer au mieux
	\newImage[H]{2}{test.png}{ceci est un test}{test} : Forcer le placement de l'image à cet endroit exact.
\end{comment}
\newcommand{\newImage}[5][ht]
{
\begin{figure}[#1]
    \centering
    \includegraphics[scale=#2]{img/#3}
    \caption{#4}
    \label{fig:#5}
\end{figure}
}

\begin{comment}
Fonction utilisée pour insérer un emoji. 
Paramètres : 
#1 : Emoji souhaité, au format unicode 
Exemple d'utilisation :
\emoji{😊}
\end{comment}
\newcommand{\emoji}[1]
{
	{\DejaSans #1}
}

%%%%%%%%%%%%%%%%%%%%%%%%%%%%%%%%%%%%%%%%%%%%%%%%%%%%%%%%%%%%%%%%%%%%%%%%%%%%%%%
% Same as before, but annexe without captions.
\newcommand{\newImageAnnexe}[5][ht]
{
\begin{figure}[#1]
    \centering
    \includegraphics[scale=#2]{img/#3}
    \caption{#5}
    \label{annexe:#4}
\end{figure}
}

%%%%%%%%%%%%%%%%%%%%%%%%%%%%%%%%%%%%%%%%%%%%%%%%%%%%%%%%%%%%%%%%%%%%%%%%%%%%
%\begin{comment}
%	Fonction utilisée pour ajouter une URL.
%	@see http://timmurphy.org/2010/04/04/referencing-website-urls-with-latex-bibtex/ ?
%	Paramètres : 
%	#1 : url (doit OBLIGATOIREMENT faire référence à un id de main.bib)
%	#2 : Nom donné
%	Exemple d'utilisation :
%	\addURL{www.google.fr}{Google}
%\end{comment}
%\newcommand{\addURL}[2]
%{
%#2\footnote{#1 (cf. entrée \cite{#1} des références)}
%}

%%%%%%%%%%%%%%%%%%%%%%%%%%%%%%%%%%%%%%%%%%%%%%%%%%%%%%%%%%%%%%%%%%%%%%%%%%%%%
\begin{comment}
	Fonction pour ajouter un acronyme contenant une définition au glossaire.
	Paramètres : 
	#1 : id acronyme
	#2 : Nom acronyme court
	#3 : Nom complet acronyme
	#4 : Description dans le glossaire

	Exemple d'utilisation :
	\newAcronym{API}{API}{Application Programming Interface}{Définition de API}
\end{comment}
\newcommand{\newAcronym}[4]
{
	\newglossaryentry{#1-g}
	{
		name={#3},
		description={#4},
		first={\glsentrylong{#1} (#2)},
		long={#3}
	}

	\newglossaryentry{#1}{
		type=\acronymtype,
		name={#2},
		description={#3},
		first={\glsentrylong{#1} (#2)\glsadd{#1-g}},
		long={#3},
		see=[Voir glossaire:]{#1-g}
	}
}

%%%%%%%%%%%%%%%%%%%%%%%%%%%%%%%%%%%%%%%%%%%%%%%%%%%%%%%%%%%%%%%%%%%%%%%%%%%%%%%%%%%%%%%%%%%%%%
\begin{comment}
	Quelques commandes utiles pour la compilation avec un index ou des références.
	
	Build Glossaries
		- "makeindex.exe -s main.ist -t main.glg -o main.gls main.glo"
	
	Build References
	- "bibtex main"
	
	Exemple bibliographie dans le fichier main.bib :
	\@Misc{ 
		google, % Identifiant pour y faire références
		note = {}, % Une note optionnelle
		title = {Google}, % Titre
		author = {\url{google.com}} % Lien vers URL ou titre du livre.
	}
	
	Créer des tableaux LateX facilement
	http://www.tablesgenerator.com
\end{comment}

%%%%%%%%%%%%%%%%%%%%%%%%%%%%%%%%%%%%%%%%%%%%%%%%%%%%%%%%%%%%%%%%%%%%%%%%%%%%%%%%%%%%%%%%%
% This is where the magic happens..
%\newcommand{\nocontentsline}[3]{}
%\newcommand{\tocless}[2]{\bgroup\let\addcontentsline=\nocontentsline#1{#2}\egroup}