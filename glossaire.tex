% Entrées du glossaire + acronymes

\begin{comment}
Glossaire : 
  Nouvelle entrée :
    \newglossaryentry{test} % l'id référençant cette entrée
    {%
        name={test}, % le terme à référencer (l'entrée qui apparaîtra dans le glossaire)
        description={<description>}, % la description du terme (sans retour à la ligne),
        plural={tests} % au pluriel
    }
  Usage : 
    \gls{test} Retourne "test"
    \Gls{test} Retourne "Test"
    \glsplural{test} Retourne "tests"
    \Glsplural{test} Retourne "Tests"
        
Pour des mots utilisés fréquemment, petite astuce : 
	\newcommand{\bsr}{bilan scientifique régional} 
Ensuite, dans le contenu, il suffit d'y faire référence comme ceci : "\bsr{}"
\end{comment}

%%%%%%%%%%%%% TERMES DU GLOSSAIRE %%%%%%%%%%%%%%%

\newglossaryentry{framework}
{%
    name={framework},
    description={(Anglicisme informatique) Ensemble de composants structurels permettant de construire des logiciels ou des sites Internet}
}

\newglossaryentry{cycle_v}
{%
	name={Cycle en V},
	description={TODO définition}
}

\newglossaryentry{earlyadopter}
{%
	name={Early Adopter},
	description={primo adoptant},
}

\newglossaryentry{persona}
{
	name={persona},
	plural={personas},
	description={Persona est une personne fictive dotée d'attributs et de caractéristiques sociales et psychologiques et qui représente un groupe cible. }
}

\newglossaryentry{gantt}
{
	name={diagramme de Gantt},
	plural={diagrammes de Gantt},
	description={TODO définition}
}

\newglossaryentry{timetomarket}
{
	name={time to market},
	description={TODO définition}
}

\newglossaryentry{markdown}{
	name={markdown},
	description={ le markdown est un langage de balisage léger. Son but est d'offrir une syntaxe facile à lire et à écrire.}
}

\newglossaryentry{git}{
	name={git},
	description={TODO définition}
}

\newglossaryentry{conteneur}{
	name={conteneur},
	plural={conteneurs},
	description={TODO définition}
}

\newglossaryentry{symfony}{
	name={Symfony},
	description={Framework PHP}
}

\newglossaryentry{drupal}{
	name={Drupal},
	description={CMS développé en PHP}
}

\newglossaryentry{angularjs}{
	name={AngularJS},
	description={Framework Front-end Javascript}
}

\newglossaryentry{backoffice}{
name={back-office},
description={TODO définition}
}

\newglossaryentry{frontend}{
	name={front-end},
	description={TODO définition}
}

\newglossaryentry{kickoff}{
	name={kick-off},
	description={TODO définition}
}

\newglossaryentry{mock}{
	name={mock},
	description={TODO définition mock}
}

\newglossaryentry{webhook}{
	name={webhook},
	description={TODO définition webhook}
}


%%%%%%%%%%%%% ACRONYMES %%%%%%%%%%%%%%%%%%%%%%%%%%%%%%%
\newAcronym{HTML}{HTML}{Hypertext Markup Language}{L’Hypertext Markup Language, généralement abrégé HTML, est le format de données conçu pour représenter les pages web}		
\newAcronym{KPI}{KPI}{Key Process Indicator}{TODO définition}
\newAcronym{esn}{ESN}{Entreprise au Service du Numérique}{TODO définition}
\newAcronym{SAS}{SAS}{Société par Actions Simplifiée}{TODO définition}
\newAcronym{DSI}{DSI}{Direction des Systèmes d'Information}{TODO définition}
\newAcronym{cms}{CMS}{Content Management System}{TODO définition}
\newAcronym{cnil}{CNIL}{Commission Nationale Informatique et Libertés}{TODO définition}
\newAcronym{LDAP}{LDAP}{Lightweight Directory Access Protocol}{TODO définition}
\newAcronym{SQL}{SQL}{Structured Query Language}{TODO définition}
\newAcronym{LAMP}{LAMP}{Linux, Apache, MySQL, PHP}{TODO définition}
\newAcronym{vm}{VM}{Virtual Machine}{Machine virtuelle, TODO définition}
\newAcronym{irc}{IRC}{Internet Relay Chat}{TODO définition}
\newAcronym{ftp}{FTP}{File Tranfert Protocol}{TODO définition}
\newAcronym{mvp}{MVP}{Minimum Viable Product}{TODO définition}
\newAcronym{vtc}{VTC}{Voitures transport avec chauffeur}{TODO définition}
\newAcronym{SLA}{SLA}{Service Level Agreement}{TODO définition}
\newAcronym{PHP}{PHP}{PHP Hypertext Preprocessor}{TODO définition}
\newAcronym{DNS}{DNS}{Domain Name Server}{TODO définition}
\newAcronym{RGAA}{RGAA}{Référentiel Général d'Accessibilité pour les Administrations}{TODO définition}
\newAcronym{RGPD}{RGPD}{Règlement général sur la protection des données}{TODO définition}
\newAcronym{API}{API}{Application Programming Interface}{TODO définition}
\newAcronym{UX}{UX}{User eXperience}{TODO définition}
\newAcronym{SSH}{SSH}{Secure SHell}{TODO définition}
\newAcronym{SFTP}{SFTP}{Secure File Transfert Protocol}{TODO définition}
\newAcronym{PCA}{PCA}{Plan Continuité Activité}{TODO définition PCA}
\newAcronym{PRA}{PRA}{Plan Reprise Activité}{TODO définition PRA}
\newAcronym{CAF}{CAF}{Caisse d'Allocation Familiale}{TODO}
\newAcronym{STD}{STD}{Spécifications Techniques Détaillées}{TODO Définition STD}
\newAcronym{SFD}{SFD}{Spécifications Fonctionnelles Détaillées}{TODO Défintion SFD}
\newAcronym{E2E}{E2E}{End 2 End}{TODO Défintion E2E}

%%%%%%%%%%%% MOT UTILISES FREQUEMMENT / VARIABLES %%%%%%%%%%%%
\newcommand{\bv}{Bureau Veritas}
\newcommand{\na}{Nouvelle-Aquitaine}
\newcommand{\onepoint}{onepoint}
\newcommand{\naq}{Nouvelle-Aquitaine}
\newcommand{\problematique}{Comment l'automatisation peut-elle permettre d'améliorer le cycle de vie d'une application ?}
\newcommand{\etsy}{Etsy}
\newcommand{\github}{Github}
