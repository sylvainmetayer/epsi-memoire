% Entrées du glossaire + acronymes

\begin{comment}
Glossaire : 
  Nouvelle entrée :
    \newglossaryentry{test} % l'id référençant cette entrée
    {%
        name={test}, % le terme à référencer (l'entrée qui apparaîtra dans le glossaire)
        description={<description>}, % la description du terme (sans retour à la ligne),
        plural={tests} % au pluriel
    }
  Usage : 
    \gls{test} Retourne "test"
    \Gls{test} Retourne "Test"
    \glsplural{test} Retourne "tests"
    \Glsplural{test} Retourne "Tests"
        
Pour des mots utilisés fréquemment, petite astuce : 
	\newcommand{\bsr}{bilan scientifique régional} 
Ensuite, dans le contenu, il suffit d'y faire référence comme ceci : "\bsr{}"
\end{comment}

%%%%%%%%%%%%% TERMES DU GLOSSAIRE %%%%%%%%%%%%%%%

\newglossaryentry{framework}
{%
    name={framework},
    description={(Anglicisme) "Boite à outil" logicielle permettant de construire des logiciels ou des sites Internet selon des conventions et des normes définies}
}

\newglossaryentry{cycle_v}
{%
	name={Cycle en V},
	description={Modèle d'organisation se définissant deux axes principaux distincts : le détail des spécifications et leur réalisation et l'axe de validation, afin de vérifier que le produit est conforme aux spécification.}
}

\newglossaryentry{earlyadopter}
{%
	name={early adopter},
	plural={early adopters},
	description={primo adoptant. il s'agit de l'un des premiers utilisateurs d'un application. Ces derniers sont alors souvent utilisés pour effectuer des retours sur l'état et les fonctionnalités de l'application.},
}

\newglossaryentry{persona}
{
	name={persona},
	plural={personas},
	description={Un persona est une personne fictive dotée d'attributs et de caractéristiques sociales et psychologiques et qui représente un groupe cible.}
}

\newglossaryentry{gantt}
{
	name={diagramme de Gantt},
	plural={diagrammes de Gantt},
	description={Diagramme de modélisation des différentes tâches à effectuer sur un projet, leur pré-requis (dépendance de tâche) via une réprésentation graphique}
}

\newglossaryentry{timetomarket}
{
	name={time to market},
	description={Délai entre la naissance/conception d'une idée et sa mise sur le marché.}
}

\newglossaryentry{markdown}{
	name={markdown},
	description={Langage de balisage léger dont le but est d'offrir une syntaxe facile à lire et à écrire.}
}

\newglossaryentry{conteneur}{
	name={conteneur},
	plural={conteneurs},
	description={Empaquetage d'une application et de ses dépendances dans un environnement isolé. A la différence d'une machine virtuelle qui inclut un système d'exploitation entier, le conteneur s'exécute sur la machine hôte, de façon isolée et en partageant les ressources physiques de l'hôte}
}

\newglossaryentry{symfony}{
	name={Symfony},
	description={Framework PHP édité par la société SensioLabs}
}

\newglossaryentry{drupal}{
	name={Drupal},
	description={Système de gestion de contenu développé en PHP et utilisant des composants de Symfony depuis sa version 8}
}

\newglossaryentry{angularjs}{
	name={AngularJS},
	description={Framework Front-end Javascript développé par Google.}
}

\newglossaryentry{backoffice}{
name={back-office},
description={Partie d'une application web non visible du public, dédiée aux administrateurs/rédacteur d'un site, destinée à administrer et gérer les fonctionnalités et/ou contenus de ce dernier}
}

\newglossaryentry{frontend}{
	name={front-end},
	description={A la différence du back-office, le front-end ou front-office est la partie visible d'un site par les utilisateurs finaux. C'est via le front-office que les utilisateurs vont accéder et interagir avec le site.}
}

\newglossaryentry{kickoff}{
	name={kick-off},
	description={(Angliscisme) Pouvant être traduit littéralement comme une "réunion de départ [d'un projet]", le kick-off est la première réunion entre le client et l'équipe projet. Elle a pour but de présenter les différentes personnes intervenant sur le projet ainsi que définir une première vision du projet (date limite, besoin\ldots)}
}

\newglossaryentry{mock}{
	name={mock},
	description={En informatique, un mock est une simulation d'une classe, afin de pouvoir tester une classe avec une autre, lorsque cette dernière à des dépendances sur des objets externes ou qui ne sont pas encore implémentés}
}

\newglossaryentry{webhook}{
	name={webhook},
	description={Un webhook est un point d'entrée d'une application permettant d'effectuer une action particulière. Par exemple, pour décléncher une action sur un système d'intégration continue, un SCM va envoyer un webhook sur une URL définie par le système d'intégration continue afin de déclencher le lancement de l'action.}
}

\newglossaryentry{test-a-b}{
	name={test A/B},
	description={Technique marketing concistant à utiliser deux versions d'une page/application/email et déterminer celle étant la plus appréciée des utilisateurs. Pour cela, une petite portion d'un groupe d'utilisateurs A va interagir avec la version A et une autre petite portion B va interagir avec la version B. Il est important que les deux versions reçoivent le même nombres d'utilisateurs pour ne pas fausser les résultats. On génère ensuite des statistiques sur les résultats obtenus et on en déduit la version préférée des utilisateurs, qui pourra alors être utilisée à une plus grande échelle.}
}

\newglossaryentry{hotfix}{
	name={hotfix},
	description={Correctif à chaud. Il s'agit d'un correctif urgent, ce qui peut arriver si un incident en production est déclaré et qu'il faut publier un correctif rapidement.}
}


\newglossaryentry{artifactory}{
	name={artifactory},
	description={Logiciel produit par Jfrog, un artifactory est un gestionnaire de fichiers binaire}
}

\newglossaryentry{code-smell}{
	name={code smell},
	description={Mauvaise pratique de développement menant le plus souvent à l'apparition de bugs.}
}

\newglossaryentry{checksum}{
	name={checksum},
	description={Calcul d'une empreinte numérique d'un fichier représentée sous forme d'une chaine de caractères. Chaque empreinte étant unique et le coût de calcul d'une empreinte étant faible, cela fournit une méthode de vérification rapide de l'intégrité d'un fichier.}
}

%%%%%%%%%%%%% ACRONYMES %%%%%%%%%%%%%%%%%%%%%%%%%%%%%%%
\newAcronym{HTML}{HTML}{Hypertext Markup Language}{L’Hypertext Markup Language, généralement abrégé HTML, est le format de données conçu pour représenter les pages web}
\newAcronym{HTTP}{HTTP}{Hypertext Transfer Protocol}{Protocole de communication permettant l'échange entre un client (le plus souvent un navigateur web) et un serveur (le plus souvent un serveur web). Le port standard utilisé pour ce protocole est le port 80.}
\newAcronym{KPI}{KPI}{Key Process Indicator}{En français ICP (indicateurs clés de performance), un KPI est un témoin utilisé pour mesurer l'efficacité d'une action.}
\newAcronym{esn}{ESN}{Entreprise au Service du Numérique}{Société de services (hébergement, conseil, réalisation, développement, maintenance, audit, formation\ldots) dans le domaine des nouvelles technologies et de l'informatique.}
\newAcronym{cms}{CMS}{Content Management System}{En français SGC (système de gestion de contenu), un CMS est un logiciel destiné à la publication et mise à jour de contenu sur un site web dynamique. Un CMS permet souvent de gérer des types de contenu (article, FAQ\ldots) et définit des rôles (administrateur, rédacteurs, relecteurs\ldots) afin que plusieurs personnes puissent accéder aux parties à l'application, selon leur besoins.}
\newAcronym{cnil}{CNIL}{Commission Nationale Informatique et Libertés}{Autorité administrative indépendante française chargée de vérifier que l'informatique soit au service de chacun et ne porte pas atteinte à une personne, à ses droits, ses libertés ou à sa vie privée.}
\newAcronym{LDAP}{LDAP}{Lightweight Directory Access Protocol}{Norme définissant la structure d'un annuaire informatique via une représentation sous forme d'arbre.}
\newAcronym{SQL}{SQL}{Structured Query Language}{Langage permettant d'effectuer des requêtes sur des bases de données relationnelles. Il est ainsi possible de consulter, modifier, ajouter ou supprimer les données présentes sur ces dernières et d'en définir la structure.}
\newAcronym{LAMP}{LAMP}{Linux, Apache, MySQL, PHP}{Acronyme définissant un ensemble de logiciels libres utilisés pour construire un serveur Web. L'acronyme provient des logiciels utilisés : Linux, le système d'exploitation, Apache, le serveur Web, MySQL, la base de données relationnelle et PHP, l'interpréteur de script.}
\newAcronym{vm}{VM}{Virtual Machine}{Émulation complète d'un système informatique, de ses ressources physiques (disques, RAM, processeur, système d'exploitation) jusqu'à ses fichiers. La machine virtuelle est exécutée de façon isolée sur la machine hôte.}
\newAcronym{irc}{IRC}{Internet Relay Chat}{Protocole de communication textuel instantané utilisé sur Internet}
\newAcronym{ftp}{FTP}{File Tranfert Protocol}{Protocole de communication permettant l'échange de fichiers entre deux machines.}
\newAcronym{mvp}{MVP}{Minimum Viable Product}{En français produit minimum viable, cela définit l'application la plus simple possible répondant au besoin primaire de cette dernière. Cela permet donc d'itérer rapidement afin de fournir une solution à ce besoin et permet de rapidement tester et faire évoluer l'application de façon incrémentale.}
\newAcronym{SLA}{SLA}{Service Level Agreement}{Document composé d'un ensemble de clauses définissant la norme de service attendu un client et une entreprise.}
\newAcronym{PHP}{PHP}{PHP Hypertext Preprocessor}{TODO définition}
\newAcronym{DNS}{DNS}{Domain Name Server}{Langage de script utilisé pour la conception de sites web dynamiques.}
\newAcronym{RGAA}{RGAA}{Référentiel Général d'Accessibilité pour les Administrations}{TODO définition}
\newAcronym{RGPD}{RGPD}{Règlement général sur la protection des données}{Référentiel des standards d'accessibilités français}
\newAcronym{API}{API}{Application Programming Interface}{Ensemble de classes ou méthodes exposé publiquement pour être utilisés par d'autres services}
\newAcronym{UX}{UX}{User eXperience}{Définit la qualité du parcours utilisateurs lors de l'utilisation d'un logiciel}
\newAcronym{SSH}{SSH}{Secure SHell}{Protocole de communication sécurisée}
\newAcronym{SFTP}{SFTP}{Secure File Transfert Protocol}{Permet l'échange de fichier via l'utilisation du protocole SSH}
\newAcronym{PCA}{PCA}{Plan Continuité Activité}{Document décrivant les actions à effectuer après un sinistre afin de maintenir l'activité de l'entreprise}
\newAcronym{PRA}{PRA}{Plan Reprise Activité}{Document décrivant les actions à entreprendre pour relancer et remettre en route un système d'information suite à un incident majeur.}
\newAcronym{DMZ}{DMZ}{Demilitarized Zone}{Réseau contenant des machines isolées du réseau local et qui sont sujet à être accédée via Internet}
\newAcronym{VPN}{VPN}{Virtual Private Network}{En français Réseau Privé Virtuel, un VPN est un système permettant de créer un lien virtuel entre des machines distantes, permettant ainsi de les isoler de façon sécurisées.}
\newAcronym{CFTL}{CFTL}{Comité Français des Tests Logiciels}{Association fondée en 2004 ayant pour but de promouvoir le métier du test logiciels}
\newAcronym{IaC}{IaC}{Infrastructure as Code}{Type d'infrastructure dans lequel les configurations des machines sont stockées sous forme de fichiers texte avant d'être exécuté par un outil pour déployer les infrastructures.}
\newAcronym{SCM}{SCM}{Software Configuration Management}{Permet de stocker et suivre les modifications d'un ensemble de fichier de façon décentralisée}
\newAcronym{SDK}{SDK}{Software Development kit}{Kit de développement logiciel facilitant le développement d'application en fournissant des outils prêt à l'emploi}
\newAcronym{IDE}{IDE}{Integrated Development Environnement}{environnement de développement intégré contenant un ensemble d'outil afin de faciliter les phases de développement et augmenter la productivité des développeurs.}
\newAcronym{ROI}{ROI}{Return on Investement}{Ratio financier prenant en compte les revenus générés par rapport à la somme investie initialement et permettant de déterminer la rentabilité d'une tâche}
\newAcronym{CTO}{CTO}{Chief Technical Officer}{En français Directeur des Nouvelles Technologies, c'est une personne en charge de l'innovation technique. Il va ainsi étudier les nouvelles technologies et indiquer leur potentiel au sein de l'entreprise et suivre leur mise en place au sein de l'entreprise.}
\newAcronym{OWASP}{OWASP}{Open Web Application Security Project}{Organisation travaillant sur la sécurité des applications Web ayant pour but de publier des recommandations sur les bonnes pratiques de sécurité à intégrer au sein des applications web.}

%%%%%%%%%%%% MOT UTILISES FREQUEMMENT / VARIABLES %%%%%%%%%%%%
\newcommand{\bv}{Bureau Veritas}
\newcommand{\na}{Nouvelle-Aquitaine}
\newcommand{\onepoint}{onepoint}
\newcommand{\naq}{Nouvelle-Aquitaine}
\newcommand{\problematique}{Comment l'automatisation peut-elle permettre d'améliorer le cycle de vie d'une application ?}
\newcommand{\etsy}{Etsy}
\newcommand{\github}{GitHub}
\newcommand{\devops}{DevOps}
\newcommand{\dev}{Dev}
\newcommand{\ops}{Ops}

