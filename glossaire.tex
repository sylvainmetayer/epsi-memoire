% Entrées du glossaire + acronymes

\begin{comment}
Glossaire : 
  Nouvelle entrée :
    \newglossaryentry{test} % l'id référençant cette entrée
    {%
        name={test}, % le terme à référencer (l'entrée qui apparaîtra dans le glossaire)
        description={<description>}, % la description du terme (sans retour à la ligne),
        plural={tests} % au pluriel
    }
  Usage : 
    \gls{test} Retourne "test"
    \Gls{test} Retourne "Test"
    \glsplural{test} Retourne "tests"
    \Glsplural{test} Retourne "Tests"
        
Pour des mots utilisés fréquemment, petite astuce : 
	\newcommand{\bsr}{bilan scientifique régional} 
Ensuite, dans le contenu, il suffit d'y faire référence comme ceci : "\bsr{}"
\end{comment}

%%%%%%%%%%%%% TERMES DU GLOSSAIRE %%%%%%%%%%%%%%%

\newglossaryentry{framework}
{%
    name={framework},
    description={(Anglicisme informatique) Ensemble de composants structurels permettant de construire des logiciels ou des sites internet}
}

\newglossaryentry{cycle_v}
{%
	name={Cycle en V},
	description={TODO}
}

%%%%%%%%%%%%% ACRONYMES %%%%%%%%%%%%%%%%%%%%%%%%%%%%%%%
\newAcronym{HTML}{HTML}{Hypertext Markup Language}{L’Hypertext Markup Language, généralement abrégé HTML, est le format de données conçu pour représenter les pages web}		
\newAcronym{KPI}{KPI}{Key Process Indicator}{TODO}
\newAcronym{esn}{ESN}{Entreprise au Service du Numérique}{TODO}
\newAcronym{SAS}{SAS}{Société par Actions Simplifiée}{TODO}
\newAcronym{DSI}{DSI}{Direction des Systèmes d'Information}{TODO}
\newAcronym{cms}{CMS}{Content Management System}{TODO}

%%%%%%%%%%%% MOT UTILISES FREQUEMMENT / VARIABLES %%%%%%%%%%%%
\newcommand{\bv}{Bureau Veritas}
\newcommand{\na}{Nouvelle-Aquitaine}
\newcommand{\onepoint}{onepoint}
\newcommand{\naq}{Nouvelle-Aquitaine}
\newcommand{\problematique}{Comment l'automatisation peut-elle permettre d'améliorer le cycle de vie d'une application ?}
\newcommand{\drupal}{Drupal}