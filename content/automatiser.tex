\todo{Automatiser un projet}

\subsection{Partie Management, gestion de projet}

Pré-requis, comment gérer la mise en place avec le management, ... Gérer les différentes personnes impliquées, ...

\subsection{Localement, lors des phases de développement}

But : prévenir l'erreur avant qu'elle ne quitte le poste du développeur (et soit donc sur le dépôt Git).

 \begin{itemize}
 	\item Linter
 	\item Docker, stack reproductible et indépendante selon environnement (windows/linux/mac) 
 	\item IDE à configurer, éventuellement partager la configuration (ex: PHP Storm Code Style)
 	\item Makefile / scripts shell de taches récurrentes (cache clear, installation, ...)
 	\item Git hook pour vérifier que tout est ok.
 \end{itemize}

\subsection{L'importance des tests}

Tout l'automatisation ne sert à rien sans une bonne campagne de tests, fonctionnel, unitaire, d'intégration, E2E, ...

Parler de Protractor et Squash sur \bv. Tests automatisés unitaires (phpunit) permettant à la plateforme d'intégration continue de vérifier la conformité du code par rapport aux attentes métier.

différence entre intégration continue et tests unitaire (image du lavabo + séchoir à mains côte à côte). Les test unitaires vont tester la cohérence d'une partie du système tandis que l'intégration va tester la cohérence de l'échange entre divers systèmes / fonctions.

\subsection{Intégration et déploiement continu}

Retour le plus rapide possible pour éviter des bugs en prod. 

Vérifier les dépendances (failles, ...)

le reporting de bug automatique (sentry.io)

Intégration des tests dans jenkins

Monitoring : monit pour monitorer l'état d'application ? (perso)

Jenkins, Travis, Déployer automatiquement à partir d'un push sur dev, en recette, intégration, staging, et lorsque tout passe, en production.

\textit{Exemple : parler de \clubSportif Automatisation de la création de la base de données (projet club sportif) et de la restauration des données}

\subsection{Et la sécurité dans tout ça ?}

Gestion des tokens, gestion des comptes.
Un compte admin ou plusieurs sous compte avec des droits limités ? En cas d'intrusion, un attaquant a moins de chance de corrompre tout le système.
Infra : restriction d'accès (htaccess / firewall / ..?)


Attention à l'endroit ou sont fait les builds/tests. Il est préférable, pour des raisons de sécurité évidente d'utiliser un outil auto-hébergé, open source, plutôt que des outils clés en main, ou les données sont envoyées sur des serveurs inconnus. 

Dans certains cas, l'application ne doit pas quitter le périmètre de l'entreprise,il faut donc avoir la maitrise totale de la chaine d'industrialisation