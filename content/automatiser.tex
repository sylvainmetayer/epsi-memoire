\subsection{Pré-requis à l'automatisation}

L'automatisation ne peut être utilisée, ou sera risquée, ou plus à même d'amener des régression si de mauvaise pratique sont présentes.

\begin{itemize}
	\item variable codée en dur à la place de variable d'environnement
	\item ...
\end{itemize}

De plus, il faut une certaine organisation.

Cela peut-être perçu comme une perte de temps par certaines personnes (hiérarchie, manager, ...), qui y verrons la une perte d'argent et de temps par exemple. Il faut alors pouvoir montrer que cela est rentable, au travers de \gls{KPI} bien déterminé.

Le fait d'automatiser des processus permet de gagner du temps, et par conséquent de l'argent et de consacrer ses efforts à d'autres taches qui peuvent apporter de la valeur métier.

Le coût horaire libéré, calculer à partir de combien de temps il est rentable.

Par exemple, une tache à 3000€ qu'on automatise et qui ne coute plus que 300€ est rentable en 10 semaines.

Facteur humain : Le temps libéré par l'automatisation des tâches peut permettre de souder les liens d'une équipe et d'améliorer les relations de cette dernière. Cela libère du temps pour du team building par exemple.

Un sujet technique qui rapproche en terme d'humain

Idées de \gls{KPI} en vracs.

Métrique - KPI - temps de déploiement / nombre d'incident / uptime / Nombre de build KO / Nombre de build OK ... permettent de définir l'impact des services mis en place sur le S.I

\begin{itemize}
	\item temps de déploiement
	\item taux de déploiement succès
	\item couverture de code
	\item tests au verts
	\item derniers build KO
\end{itemize}

PArler de l'importance de l'implication du client. 

Ex :  Rédaction de SFD, qui évolue tout les 4 matins, et demande un changement dans l'architecture => automatisation perdante.


\subsection{Localement, lors des phases de développement}

But : prévenir l'erreur avant qu'elle ne quitte le poste du développeur (et soit donc sur le dépôt Git).

 \begin{itemize}
 	\item Linter
 	\item Docker, stack reproductible et indépendante selon environnement (windows/linux/mac) 
 	\item IDE à configurer, éventuellement partager la configuration (ex: PHP Storm Code Style)
 	\item Makefile / scripts shell de taches récurrentes (cache clear, installation, ...)
 	\item Git hook pour vérifier que tout est ok.
 \end{itemize}

\subsection{Intégration continue}

Retour le plus rapide possible pour éviter des bugs en prod. 

Vérifier les dépendances (failles, ...)

le reporting de bug automatique (sentry.io)

Intégration des tests dans gitlab-ci / jenkins

Monitoring : monit pour monitorer l'état d'application ? (perso)

\subsection{Déploiement continu}

Jenkins, Travis, Déployer automatiquement à partir d'un push sur dev, en recette, intégration, staging, et lorsque tout passe, en production.

\todo{parler de \clubSportif Automatisation de la création de la base de données (projet club sportif) et de la restauration des données}

\subsection{L'importance des tests}

Tout l'automatisation ne sert à rien sans une bonne campagne de tests, fonctionnel, unitaire, d'intégration, E2E, ...

Parler de Protractor et Squash sur \bv. Tests automatisés unitaires (phpunit) permettant à la plateforme d'intégration continue de vérifier la conformité du code par rapport aux attentes métier.

différence entre intégration continue et tests unitaire (image du lavabo + séchoir à mains côte à côte). Les test unitaires vont tester la cohérence d'une partie du système tandis que l'intégration va tester la cohérence de l'échange entre divers systèmes / fonctions.

\subsection{Et la sécurité dans tout ça ?}

Gestion des tokens, gestion des comptes.
Un compte admin ou plusieurs sous compte avec des droits limités ? En cas d'intrusion, un attaquant a moins de chance de corrompre tout le système.
Infra : restricition d'accès (htaccess / firewall / ..?)
