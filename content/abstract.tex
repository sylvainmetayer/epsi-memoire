% !TeX spellcheck = en_GB

\begin{otherlanguage}{english}
\begin{abstract}

{\em \large
	This document is made with \LaTeX. The latest version of this document, along with its source code, can be found at the following website.

	\begin{center}
		\url{https://memoire.epsi.sylvainmetayer.fr}
	\end{center}

	Every word followed by the \frquote{*} symbol indicates a reference to the glossary available at page \pageref{glossaire}. Every number surrounded by brackets \frquote{[1]} indicates a reference to an element of the bibliography available at page \pageref{bibliographies}
}

\hrulefill

\begin{large}

The goal of this document is to show you the advantages and tools to implement a \devops{} and automation approach inside an informatic project. Despite the fact that this document has a technical part, it is meant to be read by all people wishing to read it.

This document will start with an history of automation, to install a context of this document. It will then detail each part inside the life-cycle of an application. This will allow the reader to have an overview of the life-cycle of an application.

We will the talk about a practical case, demonstrating the impact of implementing a \devops{} and automation approach with one company, \etsy. This will show the tools and changes made by the company to perform its transition to a \devops{} approach and set up automation processes. We will also talk about the difficulty to set up such a system and why deployments or developments became more complex over time.

In the end, we will get into the concrete part of the subject with the way to set up automation approach, from different perspective : organizational, financial, technical alongs with the prerequisite of such an approach

\end{large}

\end{abstract}
\end{otherlanguage}
