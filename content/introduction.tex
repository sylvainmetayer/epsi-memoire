\subsection{Accroche}
L'automatisation est souvent perçue comme un moyen de gagner en productivité, temps, et donc de rendre des projets toujours plus rentables.
	
Les projets informatiques sont de plus en plus nombreux, que cela soit des logiciels de bureau, des applications web, ou encore avec les nouveaux terminaux, des applications mobile, tablettes ou même pour montres connectées.
	
Combien de projet sont encore déployés manuellement car aucune automatisation n'est présente sur ce dernier ? 
	
En plus d'une perte de temps, parfois importante, cela engendre un stress chez les développeurs, qui à chaque livraison redoute les régressions qui pourraient survenir ou encore les bugs de déploiement.

L'automatisation peut également permettre d'améliorer l'arrivée d'un nouveau développeur sur un projet. Il n'est en effet par rare de voir des projets ou la configuration de l'environnement requiert à elle seule plusieurs jours, sans que le développeur puisse vraiment commencer à travailler. En plus d'être frustrant pour le développeur qui a l'impression de ne pas avancer, cela n'apporte pas de valeur ajoutée ou de connaissance sur le projet.
	
L'automatisation va permettre d'améliorer la fiabilité ainsi que la confiance des développeurs et clients dans le projet, puisque des tests automatisés ainsi qu'une chaine d'industrialisation complètement automatisée permet ainsi de déployer avec confiance une application.
	
Nous allons donc tenter de répondre à la problématique suivante :

\todo[color=cyan]{Terminer l'accroche}
	
{\LARGE \problematique}

\subsection{Définition}

Avant de continuer, il convient de s'attarder sur ce qu'est l'automatisation.

Selon le Larousse, l'automatisation est le 

\begin{quote}
fait d'automatiser l'exécution d'une tâche, d'une suite d'opérations...
\end{quote}

\todo[color=red]{Définir l'automatisation}

\subsection{Historique}

Idées : 
	
- Mode opératoire suivi religieusement, 

- script expect

- Comment déployait-on avant ?

-  ...

\todo[color=cyan]{Faire un historique de l'automatisation}

\subsection{Que peut-on automatiser ?}

\todo[color=cyan]{Que peut-on automatiser ?}

\subsection{Pourquoi automatiser ?}

Quels en sont les avantages ?

- éviter erreurs développeurs

- Permettre un suivi de l'évolution du projet via les PR + issues, permet de suivre les demandes et leur résolutions => demande traitée automatiquement déployée, suivi de l'activité de chaque dev pour le CRA\ldots

- éviter script executé avec mauvais paramètres

\todo[color=blue]{Pourquoi automatiser ?}

\subsection{Présentation entreprise \& mission}

L'entreprise dans laquelle j'effectue mon alternance depuis septembre 2017 est \onepoint. 

\xmakefirstuc{\onepoint{}} est une \gls{esn} à taille humaine. Son domaine d'activité est d'accompagner ses clients dans leur transformation numérique, 

C'est une \gls{SAS} disposant de 14 implantations dans le monde. L'entreprise a effectué en 2018 un chiffre d'affaires de 300 million d'euros.

Elle dispose de 2300 collaborateurs, en moyenne agé de 33 ans.

\xmakefirstuc{\onepoint{}} se compose de plusieurs communautés.

\begin{itemize}
	% Espacement entre les items.
	% \setlength\itemsep{0em}
	\item Des communautés \emph{régions}, permettant de regrouper les collaborateurs par leur proximité géographique.
	\item Des communautés \emph{expertise}, regroupant l'expertise de chacun, et permettant à tous de progresser. On y retrouve par exemple la communauté Sécurité ou encore Architecture.
	\item Des communautés \emph{support}, tel que la \gls{DSI}, ou les Ressources Humaines, nécessaire au fonctionnement de l'entreprise.
	\item Des communautés \emph{métiers}, regroupant des personnes maitrisant les aspects métiers des différents clients, ainsi que les contraintes de ces métiers. Cela peut par exemple être les métiers des Assurances, des Banques, des Télécoms... 
\end{itemize}

Ainsi, lors du développement d'un projet, toutes ces communautés sont utilisés, afin de tirer le meilleur d'entre elle et de regrouper les personnes les plus aptes à réaliser le projet.

Cela implique aussi que chaque collaborateur peut ainsi appartenir à une ou plusieurs communautés, selon ses compétences, ses expériences et sa localisation.

De plus, \onepoint{} se définit autour de 5 valeurs, définies ci-dessous.

\begin{description}
	\item[L'authenticité]
	\item[L'élégance]
	\item[L'ouverture]
	\item[L'ouverture] Que cela soit au niveau des clients, de tout horizon, ou en interne, en s'assurant que les différentes communautés soit sans frontière, pour permettre à chacun d'évoluer.
	\item[L'engagement] Afin de garantir la qualité des différents projets réalisés.
	\item[Le courage] Pour oser sortir des sentiers battus, et continuellement proposer de nouvelles solutions innovantes.
\end{description}

\subsubsection{Historique} 

\xmakefirstuc{\onepoint{}} a été créé en 2002, par David Layani.

De 2003 jusqu'en 2007, elle va s'ouvrir à l'international, avec l'ouverture de bureaux au Canada, en Chine et en Tunisie.

En 2008, elle étend sa position en France, avec l'ouverture de deux centres de production, à Bordeaux et Nantes.

En 2015, \onepoint{} continue son développement international au Luxembourg, en Belgique et en Hollande, avant de racheter VisionIT Group.

En 2018, \onepoint{} ouvre des bureaux à Lyon et en Australie, et rachète également Weave ainsi que Géronimo, acteur important dans la conception d'application mobile.

\subsubsection{Réalisations}

\todo[color=red]{Ajouter des réalisations de onepoint}

\subsubsection{Contexte de l'alternance}

Lors de mon alternance, j'ai été amené à travailler sur plusieurs sujets.

Le premier fut la maintenance et l'évolution d'une application web de certifications dans l'industrie, \bv. Cette dernière est une entreprise de certifications pour les professionnels. De ce fait, elle dispose d’un portail de vente en ligne permettant la commande de prestations pour certifier leurs activités. Ces prestations sont souvent obligatoires lors de lancement, poursuite ou reprise d'activité. Elles permettent de garantir le respect des normes imposées à ces entreprises dans le cadre de leur activité. Cette application web est décomposée en 2 parties, une application \gls{symfony} constituant l'\gls{API} consommé par le front-end. Le \gls{frontend} est lui en \gls{angularjs} et sert de \gls{backoffice} pour les collaborateurs \bv{} lors de la rédaction de contrats. Enfin, une autre application \gls{symfony} permet aux clients de \bv{} de fournir les contrats et leur permettre d'effectuer un retour sur les contrats proposés.

L'autre projet sur lequel j'ai eu l'occasion de travailler fut le développement de portail web pour la région \naq{}. Ces portails étaient demandés afin de refondre ceux existants et de permettre une meilleure communication sur les différents services et aides offertes par la région. Certains portails ont demandé une migration venant des précédentes versions, d'autres portails étaient de nouveaux portails, ne nécessitant pas de migration. Étant donné que la région avait déjà formé tous ses utilisateurs (rédacteurs, modérateurs) au \gls{cms} \gls{drupal}, il a été demandé de conserver cette même technologie. La seule différence est que les portails seront tous sous la version 8 de \gls{drupal} à l'occasion de cette refonte, à la différence des anciens portails qui étaient sous la version 7.

Projet Nouvelle Aquitaine, chaine d'industrialisation pour pouvoir permettre déploiement de multiple sites \Gls{drupal}.

\textit{Projet \bv{}, ou il y a une architecture actuelle qui n'est pas satisfaisante pour X raisons (reproductibilité...), et qui doit être changé}.

\textit{Elaborer un schéma directeur à partir d’orientations stratégiques} => Conduite de changement

\todo[color=cyan]{Contexte alternance}

\subsection{Annonce du plan}

\todo[color=cyan]{Annonce du plan}