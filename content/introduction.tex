\epigraph{\frquote{You build it, you run it}}{Werner Vogels - \gls{CTO} - Amazon}

L'automatisation est souvent perçue comme un moyen de gagner en productivité, en temps. Elle permet donc de rendre des projets toujours plus rentables financièrement. Les projets informatiques sont de plus en plus nombreux, au travers des logiciels de bureau, des applications web ou encore avec les nouveaux terminaux, des applications mobile, tablettes ou même pour montres connectées.
	
\emph{Mais pour autant, combien de projets sont encore déployés manuellement car aucune automatisation n'est présente sur ces derniers ? Combien de projet sont gérés par deux équipes : les \frquote{\dev}, développeurs qui développent leurs applications en garantissant que \frquote{cela fonctionne sur mon poste} et les \frquote{\ops}, responsables de l'infrastructure, qui déploient l'application et doivent gérer les incidents ? Comment améliorer les relations entre ces deux équipes ?}
	
En plus d'une perte de temps, parfois importante, le manque d'automatisation engendre un stress chez les \dev{} et \ops{} qui, à chaque livraison, redoutent les régressions qui pourraient survenir. L'automatisation permet donc de rassurer les équipes dans leur façon de travailler mais peut également améliorer l'arrivée d'un nouveau développeur. Il n'est en effet par rare de voir des projets où la configuration de l'environnement requiert à elle seule plusieurs jours, sans que le développeur puisse vraiment commencer à travailler. En plus d'être frustrant pour le développeur qui a l'impression de ne pas avancer, cela n'apporte pas de valeur ajoutée ou de connaissance sur le projet.

L'automatisation va donc permettre d'améliorer la fiabilité ainsi que la confiance des développeurs et clients dans le projet. Les tests automatisés et une chaine d'industrialisation complètement automatisée permettent ainsi de déployer avec confiance une application. Nous allons donc tenter de répondre à la problématique suivante dans les pages composant ce mémoire.

\hrulefill

{\large \problematique}

\hrulefill

Avant de continuer, il convient de s'attarder sur ce qu'est l'automatisation afin de définir les termes présents dans ce document. Selon le Larousse, l'automatisation est l'\frquote{exécution totale ou partielle de tâches techniques par des machines fonctionnant sans intervention humaine.}\footnote{\url{https://www.larousse.fr/dictionnaires/francais/automatisation/6753}}. 

Ainsi, l'objectif est de gagner du temps et donc de l'argent en faisant effectuer des tâches répétables de façon automatique par une machine qui se trouve être le plus souvent -- dans notre cas -- un serveur informatique. Mais cette idée d'automatisation n'est pas nouvelle. Avec l'avancée technologique, l'Homme a toujours tenté de faciliter son travail informatique en automatisant au maximum ses tâches. 

Cela a d'abord commencé par la rédaction de manuels opératoires permettant de définir les tâches à effectuer les unes après les autres. Ces derniers étaient alors suivis religieusement par la personne en charge de déployer ou d'installer l'application. Néanmoins, cette procédure ne retirait pas le facteur humain de l'opération et rendait alors instable l'installation. En effet, comment s'assurer que la personne n'a pas manqué une étape ? 

Pour tenter de palier cela, des scripts furent mis en place. On peut notamment citer \emph{expect}\footnote{\url{https://fr.wikipedia.org/wiki/Expect}} qui est un outil permettant d'automatiser des déploiements via \gls{SSH}, \gls{SFTP}\ldots{} L'objectif de cet outil était de conserver les dispositifs existants (\gls{SSH}, \gls{SFTP}\ldots) mais de supprimer l'interaction humaine nécessaire pour déployer l'application. Au lieu d'écrire un manuel opératoire indiquant qu'il fallait se connecter avec l'utilisateur, vérifier que le paquet \emph{MonApplication} était bien présent, puis redémarrer le serveur, il suffisait alors d'écrire ces instructions dans un fichier, puis de faire exécuter ce dernier à la personne en charge du déploiement. On réduit ainsi le risque d'erreur humaine. Cette solution avait néanmoins l'inconvénient d'être assez peu maintenable et pouvait rapidement se révéler verbeux. Cela avait également l'inconvénient de ne pas être testable. On pouvait difficilement s'assurer que le script fonctionnait sur le serveur une première fois, sans faire un \emph{crash-test} en espérant que tout se passe pour le mieux. Les environnements n'étant pas spécialement tous identiques, cela pouvait également entrainer des incompatibilités, nécessitant des développements spécifiques. De plus, si l'outil utilisé (\gls{SSH} par exemple) publiait une mise à jour introduisant une incompatibilité avec la version précédente, il fallait alors revoir le script de déploiement afin de l'adapter. C'est pourquoi des outils et des méthodes ont été mis en place afin de permettre une meilleure fluidité du cycle de vie de l'application. 

\emph{Pour autant, pourquoi automatiser ? Si cela fonctionne de façon manuelle, un changement est-il nécessaire? Peut-on tout automatiser ?}

Il s'agit de réduire au maximum les erreurs humaines. Il faut garder à l'esprit que l'humain est le maillon faible d'une chaine d'industrialisation. Il ne sera jamais infaillible et c'est sur ce point que la machine se démarque car elle reproduira toujours le même résultat pour une tâche donnée. De plus, garantir via des tests automatisés la conformité de l'application est un signe évident de qualité. Le but est également de faciliter la vie des développeurs, leur permettant ainsi de se concentrer sur des tâches apportant une vraie valeur métier à l'application.

Concernant ce qu'il est possible d'automatiser, il est techniquement réalisable de tout automatiser, de la simple copie de fichier jusqu'au déploiement d'une application complexe répartie sur plusieurs serveurs, avec une redondance des données. Même une opération réalisée une fois par an peut être automatisée. D'ailleurs, il est fort probable qu'une opération effectuée à une faible fréquence par une personne soit souvent une opération qui entraine une erreur puisque la personne n'est pas habituée à effectuer cette opération. Elle a alors un grand risque de faire une erreur et de passer des heures supplémentaires à corriger cette dernière. L'inverse est également vrai, puisque si l'on effectue une tâche de façon répétitive à la main, il est possible que l'on ne fasse une erreur d'inattention à force de répétition. 

Il convient cependant d'être réaliste quant aux moyens que l'on souhaite mettre en place pour pouvoir automatiser les différentes tâches à effectuer, sous peine d'entrainer de fortes dépenses pour une tâche qui n'en nécessite pas tant et dont les effets de bords sont faibles.

Avant de commencer à aborder le sujet de ce mémoire, il convient de dire quelques mots sur l'entreprise dans laquelle j'effectue mon alternance depuis septembre 2017, \onepoint{} ainsi que sur le contexte de mon alternance. \xmakefirstuc{\onepoint{}} est une \gls{esn} à taille humaine et son domaine d'activité est d'accompagner ses clients dans leur transformation numérique. C'est une SAS\footnote{Société par actions simplifiée} disposant de 14 implantations dans le monde. L'entreprise a effectué en 2018 un chiffre d'affaires de 300 millions d'euros. Elle dispose de 2300 collaborateurs âgés en moyenne de 33 ans et se compose de plusieurs communautés.

\begin{description}
	% Espacement entre les items.
	% \setlength\itemsep{0em}
	\item [Des communautés régions] permettant de regrouper les collaborateurs par leur proximité géographique.
	\item [Des communautés expertise] regroupant l'expertise de chacun, et permettant à tous de progresser. On y retrouve par exemple la communauté Sécurité ou encore Architecture.
	\item [Des communautés support] telles que la Direction des Systèmes d'Information ou les Ressources Humaines, nécessaires au fonctionnement de l'entreprise.
	\item [Des communautés métiers] regroupant des personnes maitrisant les aspects métiers des différents clients, ainsi que les contraintes de ces métiers. Cela peut par exemple être les métiers des Assurances, des Banques, des Télécoms... 
\end{description}

Ainsi, lors du développement d'un projet, toutes ces communautés sont utilisées afin de tirer le meilleur d'entre elles et de regrouper les personnes les plus aptes à réaliser le projet. Cela signifie que chaque collaborateur peut ainsi appartenir à une ou plusieurs communautés, selon ses compétences, ses expériences et sa localisation. 

Lors de mon alternance, j'ai été amené à travailler sur plusieurs sujets. Le premier fut la maintenance et l'évolution d'une application web de certifications dans l'industrie, \bv. Cette dernière est une entreprise de certifications pour les professionnels. De ce fait, elle dispose d’un portail de vente en ligne permettant la commande de prestations pour certifier leurs activités. Ces prestations sont souvent obligatoires lors de lancement, poursuite ou reprise d'activité. Elles permettent de garantir le respect des normes imposées à ces entreprises dans le cadre de leur activité. Cette application web est décomposée en deux parties, une application \gls{symfony} constituant l'\gls{API} consommé par le front-end. Le \gls{frontend} est lui en \gls{angularjs} et sert de \gls{backoffice} pour les collaborateurs \bv{} lors de la rédaction de contrats. Enfin, une autre application \gls{symfony} permet aux clients de \bv{} de fournir les contrats et leur permettre d'effectuer un retour sur les contrats proposés.

Le second projet sur lequel j'ai eu l'occasion de travailler fut le développement de portail web pour la région \naq{}. Ces portails étaient demandés afin de refondre ceux existants et de permettre une meilleure communication sur les différents services et aides offertes par la région. Certains portails ont demandé une migration venant des précédentes versions, tandis que d'autres, étant de nouveaux projets, n'ont pas nécessité cette étape. Étant donné que la région avait déjà formé tous ses utilisateurs (rédacteurs, modérateurs) au \gls{cms} \gls{drupal}, il a été demandé de conserver cette même technologie. La seule différence est que les portails seront tous sous la version 8 de \gls{drupal} à l'occasion de cette refonte, contrairement aux anciens qui étaient sous la version 7. 

Concernant les projets \naq, j'ai également eu l'occasion d'intervenir sur la chaine d'industrialisation afin de permettre un déploiement automatisé des différents portails, d'abord en interne et à terme en pré-production ainsi qu'en production. J'ai alors pu voir toute la mise en place de la chaine d'industrialisation et son automatisation.

Afin de traiter la problématique évoquée plus haut dans cette introduction, ce document est composé de 3 parties qui permettront chacune d'apporter des éléments de réponse. La première partie va tout d'abord concerner le cycle de vie d'une application afin de se familiariser avec les différentes étapes que composent un projet. Dans un deuxième temps, l'intérêt de l'automatisation sera abordé afin de justifier son utilité au travers d'un cas pratique avec une entreprise existante. Puis dans un dernier temps, la façon d'automatiser un projet sera détaillée, selon différentes phases et moment du cycle de vie du projet.

À cet instant, vous \emph{(oui, vous, lecteur)} êtes peut-être en train de vous demander pourquoi j'ai fait un tel choix de sujet. 

Je me suis intéressé au sujet de l'automatisation lors de mon arrivée à l'EPSI en 4\up{ème} année. Durant mes deux années de formation, j'ai, avec mon binôme, réalisé de nombreux projets en ayant en tête l'objectif de toujours tenter d'intégrer des bonnes pratiques ou mettre en place une intégration continue avec des tests automatisés et des déploiements automatisés, lorsque cela était possible. Lors de mon passage en 5\up{ème} année, nous avons décidé de centraliser tous nos projets réalisés à l'EPSI au sein d'une organisation \github\footnote{\url{https://github.com/EPSIBordeaux/}}. Nous avons alors loué un serveur chez OVH\footnote{\url{https://www.ovh.com/}} pendant quelques mois pour y exposer nos projets. J'ai alors souhaité avoir une première expérience concrète avec l'automatisation et ansible, en automatisant le provisionnement du serveur et des applications que nous souhaitions y installer\footnote{Si vous êtes curieux, vous pouvez voir le résultat ici - \url{https://github.com/EPSIBordeaux/ansible-deployment}}. Ce projet, bien qu'imparfait, m'a alors donné envie d'en découvrir plus sur ce sujet. Mon sujet de mémoire était donc déjà plus ou moins décidé, il ne restait plus qu'à savoir comment l'appliquer en entreprise.

Après quelques échanges avec mon tuteur en entreprise, j'ai eu l'occasion de pouvoir travailler sur la chaine d'industrialisation des projets de la \naq. Je suis encore alternant pour quelques semaine et je bénéficie donc d'un avantage particulier : pouvoir me tromper, plus que la normale. Pouvoir prendre plus de temps qu'une personne déjà en poste pour réaliser une tâche.

J'ai énormément appris durant la réalisation de ce projet, de part mes recherches où en échangeant avec des collègues pour savoir quel était le meilleur axe à prendre pour une problématique donnée. Je suis parti de connaissances théoriques sur le sujet pour aller les appliquer sur un projet concret, sans expérience préalable, ce que je n'aurais peut-être pas pu faire aussi facilement si je n'avais pas été alternant, puisque je n'aurai pas eu cette \frquote{immunité de l'erreur}.
