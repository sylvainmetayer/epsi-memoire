\subsection{Accroche}
L'automatisation a toujours été perçue comme un moyen de gagner en productivité, temps, et donc de rendre des projets toujours plus rentables.
	
Les projets informatiques sont de plus en plus nombreux, que cela soit des logiciels de bureau, des applications web, ou encore avec les nouveaux terminaux, des applications mobile, tablettes ou même pour montres connectées.
	
Les projets augmentent donc en quantité, mais augmentent-ils en qualité ? Leur fiabilité n'est en effet pas toujours optimale.
	
Combien de projet sont encore déployé manuellement car aucune automatisation n'est présente sur le projet ? 
	
En plus d'une perte de temps, parfois importante, cela engendre un stress au niveau des équipes de développeurs, qui à chaque livraison redoute les régressions qui pourraient survenir ou encore les bugs de déploiement.

L'automatisation peut également permettre d'améliorer l'arrivée d'un nouveau développeur sur un projet. Il n'est en effet par rare de voir des projets ou la configuration de l'environnement requiert à elle seule plusieurs jours, sans que le développeur puisse vraiment commencer à travailler.
	
L'automatisation va permettre d'améliorer la fiabilité ainsi que la confiance des développeurs et clients dans le projet, puisque des tests automatisés ainsi qu'une chaine d'industrialisation complètement automatisée permet ainsi de déployer avec confiance une application.
	
Nous allons donc tenter de répondre à la problématique suivante :

\todo{Terminer l'accroche}
	
{\LARGE \problematique}

\subsection{Définition}

Avant de continuer, il convient de s'attarder sur ce qu'est l'automatisation.

Selon le Larousse, l'automatisation est le 

\begin{quote}
fait d'automatiser l'exécution d'une tâche, d'une suite d'opérations...
\end{quote}

\todo{Définir l'automatisation}

\subsection{Historique}

Idées : 
	
- Mode opératoire suivi religieusement, 

- script expect

- Comment déployait-on avant ?

-  ...

\todo{Faire un historique de l'automatisation}

\subsection{Que peut-on automatiser ?}

\todo{Que peut-on automatiser ?}

\subsection{Pourquoi automatiser ?}

Quels en sont les avantages ?

- éviter erreurs développeurs

- éviter script executé avec mauvais paramètres

\todo{Pourquoi automatiser ?}

\subsection{Présentation entreprise \& mission}

L'entreprise dans laquelle j'effectue mon alternance depuis septembre 2017 est \onepoint. 

\xmakefirstuc{\onepoint{}} est une \gls{esn} à taille humaine. Son domaine d'activité est d'accompagner ses clients dans leur transformation numérique, 

C'est une \gls{SAS} disposant de 14 implantations dans le monde. L'entreprise a effectué en 2018 un chiffre d'affaires de 300 million d'euros.

Elle dispose de 2300 collaborateurs, en moyenne agé de 33 ans.

\xmakefirstuc{\onepoint{}} se compose de plusieurs communautés.

\begin{itemize}
	\item Des communautés \emph{régions}, permettant de regrouper les collaborateurs par leur proximité géographique.
	\item Des communautés \emph{expertise}, regroupant l'expertise de chacun, et permettant à tous de progresser. On y retrouve par exemple la communauté Sécurité ou encore Architecture.
	\item Des communautés \emph{support}, tel que la \gls{DSI}, ou les Ressources Humaines, nécessaire au fonctionnement de l'entreprise.
	\item Des communautés \emph{métiers}, regroupant des personnes maitrisant les aspects métiers des différents clients, ainsi que les contraintes de ces métiers. Cela peut par exemple être les métiers des Assurances, des Banques, des Télécoms... 
\end{itemize}

Ainsi, lors du développement d'un projet, toutes ces communautés sont utilisés, afin de tirer le meilleur d'entre elle et de regrouper les personnes les plus aptes à réaliser le projet.

Cela implique aussi que chaque collaborateur peut ainsi appartenir à une ou plusieurs communautés, selon ses compétences, expérience et localisation.

De plus, \onepoint{} se définit autour de 5 valeurs.

\begin{itemize}
	\item L'authenticité
	\item L'élégance
	\item L'ouverture, que cela soit au niveau des clients, de tout horizon, ou en interne, en faisant en sorte que les différentes communautés soit sans frontière, pour permettre à chacun d'évoluer
	\item L'engagement, afin de garantir la qualité des différents projets réalisés.
	\item Le courage, pour oser sortir des sentiers battus, et continuellement proposer de nouvelles solutions innovantes.
\end{itemize}

\subsubsection{Historique} 

\xmakefirstuc{\onepoint{}} a été créé en 2002, par David Layani.

De 2003 jusqu'en 2007, elle va s'ouvrir à l'international, avec l'ouverture de bureaux au Canada, en Chine et en Tunisie.

En 2008, elle étend sa position en France, avec l'ouverture de deux centres de production, à Bordeaux et Nantes.

En 2015, \onepoint{} continue son développement international au Luxembourg, en Belgique et en Hollande, avant de racheter VisionIT Group.

En 2018, \onepoint{} ouvre des bureaux à Lyon et en Australie, et rachète également Weave ainsi que Géronimo, acteur important dans la conception d'application mobile.

\subsubsection{Réalisations}

\todo{Ajouter des réalisations de onepoint}

\subsubsection{Contexte de l'alternance}

Projet Nouvelle Aquitaine, chaine d'industrialisation pour pouvoir permettre déploiement de multiple sites Drupal.

\textit{Projet \bv{}, ou il y a une architecture actuelle qui n'est pas satisfaisante pour X raisons (reproductibilité...), et qui doit être changé}.

\textit{Elaborer un schéma directeur à partir d’orientations stratégiques} => Conduite de changement

\todo{Contexte alternance}

\subsection{Annonce du plan}

\todo{Annonce du plan}