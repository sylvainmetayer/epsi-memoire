\todo{Introduction}

\todo{Quand la partie introduction sera terminée, supprimer les parties.}

\subsection{Accroche}
L'automatisation a toujours été perçue comme un moyen de gagner en productivité, temps, et donc de rendre des projets toujours plus rentable.
	
Les projets informatique sont de plus en plus nombreux, que cela soit des logiciels de bureau, des applications web, ou encore avec les nouveaux terminaux, des applications mobile, tablettes, ou même pour montres connectées.
	
Les projets augmentent donc en quantité, mais augmentent-ils en qualité? Leur fiabilité n'est en effet pas toujours optimale. 
	
Combien de projet sont encore déployé manuellement car aucune automatisation n'est présente sur le projet ? 
	
En plus d'une perte de temps, parfois importante, cela engendre un stress au niveau des équipes de développeurs, qui à chaque livraison redoute les régressions qui pourraient survenir ou encore les bugs de déploiement.

L'automatisation peut également permettre d'améliorer l'onboarding d'un nouveau développeur sur un projet. Il n'est en effet par rare de voir des projets ou la configuration de l'environnement requiert à elle seule plusieurs jours, sans que le développeur puisse vraiment commencer à travailler.
	
Également, l'automatisation va permettre d'améliorer la fiabilité et la confiance des développeurs et des clients dans le projet, puisque des tests automatisés ainsi qu'une chaine d'industrialisation complètement automatisée permet ainsi de déployer avec confiance une application.
	
Nous allons donc tenter de répondre à la question suivante : 
	
{\LARGE \problematique}

\subsection{Définition}

\subsection{Historique}

Historique Idées : 
	
\begin{itemize}
	\item mode opératoire suivi religieusement, 
	\item script expect
	\item Comment déployait-on avant ?
	\item ...
\end{itemize}

\textit{L’introduction doit remplir une fonction traditionnelle: délimiter et présenter le projet ou la mission et son contexte professionnel, annoncer les parties principales du développement. L’introduction représente environ un dixième du mémoire. Il faut absolument insister sur la bonne impression qu’elle doit donner au lecteur comme premier et décisif contact avec le mémoire.}

\subsection{Que peut-on automatiser ?}

\subsection{Pourquoi automatiser ?}

Quels en sont les avantages ?

\subsection{Présentation entreprise / mission}

\subsubsection{Contexte}

\textit{Projet \bv{}, ou il y a une architecture actuelle qui n'est pas satisfaisante pour X raisons (reproductiblité, ...), et qui doit être changé}.

\textit{Elaborer un schéma directeur à partir d’orientations stratégiques} => Conduite de changement


\subsection{Annonce du plan}