C'est maintenant la fin de ce document. Les différentes étapes du cycle de vie d'une application ont été détaillée et nous avons vu qu'il n'existait pas d'automatisation \frquote{miracle} permettant de répondre à tous les besoins mais qu'il s'agissait d'un assemblage d'outils, chacun effectuant une tâche dédiée. De plus, la diversité des projets a montré qu'il est impossible de réaliser \frquote{un outil pour les gouverner tous}.

Nous avons aussi vu que les besoins d'automatisation était de plus en plus utilisés, du fait que la complexité des applications grandit de jour en jour\footnote{bien que cela ne soit pas toujours justifié}. L'objectif est de fournir la meilleure expérience utilisateur à l'utilisateur final, tout en permettant des livraisons fréquentes et rapide afin d'apporter de la valeur ajoutée au produit.


Nous allons donc reprendre dans cette conclusion les différents avantages à mettre en place une démarche d'automatisation sur un projet.

Tout d'abord, le but est de réduire les erreurs humaines lors des phases de développement tout en permettant d'automatiser les tâches répétitives. La réduction des erreurs entraine alors une meilleure fiabilité de l'application, via la mise en place de tests automatisés. Cette confiance dans l'application permet alors de livrer des fonctionnalités plus rapidement et contribue à réduire le \gls{timetomarket} pour offrir de nouvelles fonctionnalités aux utilisateurs, ou corriger les bugs plus rapidement.

De plus, l'automatisation va également permettre une reprise d'activité plus rapide et peut donc être justifiée dans les \gls{PRA} et les \gls{PCA}, en insistant sur le fait qu'une relance rapide de l'activité peut permettre de limiter les pertes financières. Cela en dù au fait que l'automatisation permet le provisionnement de serveur de façon automatique, garantissant la reproductibilité des étapes à réaliser pour déployer une application, avec des outils tel qu'Ansible.

\todo[color=yellow]{Conclusion, 7 pages}

\subsection*{Bénéfices}

\todo[color=orange]{Benefices}

moins d'erreurs développeur, plus de fiabilité, time to market réduit, ROI avec confiance et livraison plus rapide. Relancer activité défaillante (SLA/PCA/PRA). Plus de temps pour la valeur métier. 

Documentation : arrivée sur un projet, démarrage rapide.

Scalabilité : docker - provisionnemnt rapide de novueaux serveurs avec ansible.

en dev, on souhaite des logs direct dans la console, en prod on les mets dans un fichier de log.
en dev, on veut le mode debug, en prod on le désactive.

Cela peut être perçu comme une perte de temps par certaines personnes (hiérarchie, manager...), qui y verrons la une perte d'argent et de temps par exemple. Il faut alors pouvoir montrer que cela est rentable, au travers de \gls{KPI} bien déterminés.

Le fait d'automatiser des processus permet de gagner du temps, et par conséquent de l'argent et de consacrer ses efforts à d'autres taches qui peuvent apporter de la valeur métier. Le coût horaire libéré, calculer à partir de combien de temps il est rentable.

Facteur humain : Le temps libéré par l'automatisation des tâches peut permettre de souder les liens d'une équipe et d'améliorer les relations de cette dernière. Cela libère du temps pour du team building par exemple.

Métrique - KPI - temps de déploiement / nombre d'incident / uptime / Nombre de build KO / Nombre de build OK ... permettent de définir l'impact des services mis en place sur le S.I

\begin{itemize}
	\item temps de déploiement
	\item taux de déploiement succès
	\item couverture de code
	\item tests au verts
	\item derniers build KO
\end{itemize}

\subsection*{Limites}

\todo[color=orange]{Limites}

Sur-qualité, sur-optimisation

Aucun intérêt si les tests ne sont pas fiable.

Dépend du client, de l'environnement, demande de la flexibilité, délai de mise en place, ROI

Pré-requis à l'automatisation

Attention, auto != vérité. récompenses / punition : il y a des phénommènes de triches / optimisation pour avoir le meilleur score.
% https://www.matuzo.at/blog/building-the-most-inaccessible-site-possible-with-a-perfect-lighthouse-score/

L'automatisation ne peut être utilisée, ou sera risquée, ou plus à même d'amener des régression si de mauvaise pratique sont présentes : variable codée en dur à la place de variable d'environnement

De plus, il faut une certaine organisation.

Parler de l'importance de l'implication du client. 

Ex :  Rédaction de SFD, qui évolue tout les 4 matins, et demande un changement dans l'architecture => automatisation perdante.

nécessite certains prérequis : 

- création d'environnemnet à la volée. Si on doit attendre des mois pour ça, ça n'a pas de sens. Solution => serverless avec à la demande ? 

\subsection*{Ce qui n'est pas encore automatisé}

\todo{ce qui n'est pas encore automatisé}

selon les entreprises, dépend du besoin.

\newImage[H]{0.7}{time-worth.png}{Temps pour automatiser tâches - \url{https://xkcd.com/1205/}}{time-worth}

\subsection*{Les possiblités d'automatisation dans le futur}

\todo{Parler de ce qui n'est pas encore automatisé, et les différentes pistes d'automatisation possible dans le futur.}
