\todo[color=yellow]{Conclusion, 7 pages}

- Pourquoi n'existe pas une solution unique ? => car chaque projet différent 

- Pourquoi complexité des processus de déploiement ? => car complexité application en hausse, et demande utilisateur avec toujours plus de simplicité, et donc complexe à réaliser. Passage d'une "application à la bonne franquette", à un vrai processus éprouvé pour durer

\subsection*{Bénéfices}

\todo[color=orange]{Benefices}

\textit{dentifier les leviers d’économie à actionner pour améliorer les processus Qualité.}

\begin{itemize}
	\item KPI à trouver
	\item Confiance dans la livraison
	\item ROI avec rapidité de livraison
	\item Décharge l'équipe
	\subitem Plus de temps pour de la valeur métier
	\item L'automatisation peut permettre de relancer rapidement une activité défaillante (SLA / PRA / PCA)
\end{itemize}

\begin{itemize}
	\item - Erreur développeur
	\item + de Fiabilité
	\item Preuve de qualité
	\item Time To Market réduit
	\item Reproductibilité
\end{itemize}

Comment faire en sorte que ça marche dans la durée ? Des controles / supervisons périodique afin de checker que tout va bien.

Workflow : savoir ce qui est automatisé, comment c'est mis en oeuvre ,documentation des outils, ...
Faire en sorte que l'automatisation ne casse pas et que l'on en tire quelques chose, que l'on soit nouvel arrivant sur le projet, ou développeur déjà présent sur le projet.

Scalabilité : Docker / provisionner de nouveaux serveurs rapidement avec Ansible par exemple.

Déploiement : chaine de déploiement (dev/test/inte/preprod/prod) avec chacune ses spécificités

Exemple :

en dev, on souhaite des logs direct dans la console, en prod on les mets dans un fichier de log.
en dev, on veut le mode debug, en prod on le désactive.

\subsection*{Limites}

\todo[color=yellow]{Limites}

Sur-qualité, sur-optimisation, ...

Aucun intérêt si les tests ne sont pas fiable.

Dépend du client, de l'environnement, demande de la flexibilité, délai de mise en place, ROI

Pré-requis à l'automatisation

Attention, auto != vérité \url{https://www.matuzo.at/blog/building-the-most-inaccessible-site-possible-with-a-perfect-lighthouse-score/}

L'automatisation ne peut être utilisée, ou sera risquée, ou plus à même d'amener des régression si de mauvaise pratique sont présentes.

\begin{itemize}
	\item variable codée en dur à la place de variable d'environnement
	\item ...
\end{itemize}

De plus, il faut une certaine organisation.

Cela peut-être perçu comme une perte de temps par certaines personnes (hiérarchie, manager, ...), qui y verrons la une perte d'argent et de temps par exemple. Il faut alors pouvoir montrer que cela est rentable, au travers de \gls{KPI} bien déterminé.

Le fait d'automatiser des processus permet de gagner du temps, et par conséquent de l'argent et de consacrer ses efforts à d'autres taches qui peuvent apporter de la valeur métier.

Le coût horaire libéré, calculer à partir de combien de temps il est rentable.

Par exemple, une tache à 3000€ qu'on automatise et qui ne coute plus que 300€ est rentable en 10 semaines.

Facteur humain : Le temps libéré par l'automatisation des tâches peut permettre de souder les liens d'une équipe et d'améliorer les relations de cette dernière. Cela libère du temps pour du team building par exemple.

Un sujet technique qui rapproche en terme d'humain

Idées de \gls{KPI} en vracs.

Métrique - KPI - temps de déploiement / nombre d'incident / uptime / Nombre de build KO / Nombre de build OK ... permettent de définir l'impact des services mis en place sur le S.I

\begin{itemize}
	\item temps de déploiement
	\item taux de déploiement succès
	\item couverture de code
	\item tests au verts
	\item derniers build KO
\end{itemize}

Parler de l'importance de l'implication du client. 

Ex :  Rédaction de SFD, qui évolue tout les 4 matins, et demande un changement dans l'architecture => automatisation perdante.

Dépôt git du mec qui automatise tout \url{https://github.com/NARKOZ/hacker-scripts}

nécessite certains prérequis : 



- création d'environnemnet à la volée. Si on doit attendre des mois pour ça, ça n'a pas de sens. Solution => serverless avec à la demande ? 

\subsection*{Ce qui n'est pas encore automatisé}

\subsection*{Les possiblités d'automatisation dans le futur}

\textit{Parler de ce qui n'est pas encore automatisé, et les différentes pistes d'automatisation possible dans le futur.}


