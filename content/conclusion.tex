\todo{Conclusion}
\vfill
Conclusion

Quant à la conclusion, elle doit certes d’abord résumer le mémoire et identifier les apports et  solutions du  développement,  mais  aussi,  fonction  classique, ouvrir  des  perspectives et appeler de futures contributions. Pour la première des fonctions de la conclusion, il s’agit le plus clairement possible, d’indiquer dans quelle mesure le travail et/ou la réflexion de l’apprenant-stagiaire ou alternant a abouti à des résultats concrets, opérationnels, en tout état de cause constructifs.Pour la seconde des fonctions de la conclusion, l’élargissement, le rédacteur doit savoir faire le  point  objectivement  et  ouvrir  les  pistes  de réflexions pour  résoudre  un  problème  ou améliorer des solutions. Selon le sujet, il sera utile de situer les implications possibles de la question traitée dans un domaine plus large que l’informatique.

La  conclusion  ne  doit  donc  pas  être  bâclée;  sa  dimension sera d’environ un dixième du mémoire.
\vfill
