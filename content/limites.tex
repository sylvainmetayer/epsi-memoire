\todo{Limites}

Sur-qualité, sur-optimisation, ...

Aucun intérêt si les tests ne sont pas fiable.

Dépend du client, de l'environnement, demande de la flexibilité, délai de mise en place, ROI

\subsection*{TODO}
Pré-requis à l'automatisation

L'automatisation ne peut être utilisée, ou sera risquée, ou plus à même d'amener des régression si de mauvaise pratique sont présentes.

\begin{itemize}
	\item variable codée en dur à la place de variable d'environnement
	\item ...
\end{itemize}

De plus, il faut une certaine organisation.

Cela peut-être perçu comme une perte de temps par certaines personnes (hiérarchie, manager, ...), qui y verrons la une perte d'argent et de temps par exemple. Il faut alors pouvoir montrer que cela est rentable, au travers de \gls{KPI} bien déterminé.

Le fait d'automatiser des processus permet de gagner du temps, et par conséquent de l'argent et de consacrer ses efforts à d'autres taches qui peuvent apporter de la valeur métier.

Le coût horaire libéré, calculer à partir de combien de temps il est rentable.

Par exemple, une tache à 3000€ qu'on automatise et qui ne coute plus que 300€ est rentable en 10 semaines.

Facteur humain : Le temps libéré par l'automatisation des tâches peut permettre de souder les liens d'une équipe et d'améliorer les relations de cette dernière. Cela libère du temps pour du team building par exemple.

Un sujet technique qui rapproche en terme d'humain

Idées de \gls{KPI} en vracs.

Métrique - KPI - temps de déploiement / nombre d'incident / uptime / Nombre de build KO / Nombre de build OK ... permettent de définir l'impact des services mis en place sur le S.I

\begin{itemize}
	\item temps de déploiement
	\item taux de déploiement succès
	\item couverture de code
	\item tests au verts
	\item derniers build KO
\end{itemize}

PArler de l'importance de l'implication du client. 

Ex :  Rédaction de SFD, qui évolue tout les 4 matins, et demande un changement dans l'architecture => automatisation perdante.

