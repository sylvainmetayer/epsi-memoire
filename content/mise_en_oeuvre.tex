\subsection{Cas pratique}

Maintenant que nous avons vu les différentes étapes qui composaient le cycle de vie d'une application, nous pouvons voir, au travers d'un cas pratique, la mise en place d'une démarche d'automatisation dans un projet ou, dans le cas présent, d'une entreprise.

L'entreprise retenue pour l'étude suivante est \etsy.

\todo[color=red]{Cas pratique, 5-6 pages}

Exemple de transformation : Etsy

- Vente de produit fait main

- Créé en 2003

- Au début, livraison 2 fois / semaines, mais cela durait 4h.

Besoin d'améliorer le processus pour pouvoir suivre la cadence.

Passage d'un développement lourd, non organisé à un déploiement de micro service, avec plusieurs MEP / jours.

\subsection{Difficulté d'avoir la panacée}

\todo[color=red]{Difficulté d'avoir la panacée, 1-2 pages}

- pourquoi n'existe pas une solution unique ? 

- Pourquoi complexité des processus de déploiement ? 

- chaque outil répond à un besoin, mais existe-t-il un outil all-in-one ? (non)

Solutions existantes

Chef, puppet, ansible, jenkins, travis, docker, terraform, heroku, aws, ...

Combinaison d'outils qui répondent à un besoin, pattern.


\textbf{ATTENTION : NE PAS REPONDRE A LA QUESTIOn, SEULEMENT SOULEVER DES INTERROGATIONS POUR LE LECTEUR}

Cas pratique (démo entreprise migration) / Comment le généraliser (NE PAS RÉPONDRE A CETTE QUESTION). Cas précis et généraliser, parler des autres solutions. Difficulté d'avoir une solution clé en main, une solution à tout, la panacée

Qu'est ce qui a poussé à l'automatiser ? Complexification de ces étapes (npm) pour pousser à la nécessité l'automatisation. Le fait que l'utilisateur veut toujours plus de simplicité pousse à toujours plus de complexité de réalisation. Permet de se concentrer sur des tâches à forte valeur ajoutée.

Innovation technologique 