\subsection{Cas pratique}

Maintenant que nous avons vu les différentes étapes qui composaient le cycle de vie d'une application, nous pouvons voir, au travers d'un cas pratique, la mise en place d'une démarche d'automatisation dans un projet ou, dans le cas présent, d'une entreprise.

L'entreprise retenue pour l'étude suivante est \etsy\footnote{\url{https://www.etsy.com}}. \etsy{} est une entreprise créée en 2005 spécialisée dans la vente de création personnelles. Elle permet à ses utilisateurs de créer leurs boutiques et d'y vendre leurs créations à toute la communauté. En 2013, elle comptait plus de 4 millions d'items vendus et disposait d'une communauté de 22 millions d'utilisateurs pour un total de \numprint{80000} boutiques actives sur leur site.

L'entreprise a connu une forte croissance à ses débuts et à donc du s'adapter pour pouvoir offrir une qualité de service et des évolutions constantes.

\subsection{États des lieux}

Avant de détailler la transition effectuée, faisons un état des lieux techniques avant leur transition vers l'automatisation.

\begin{itemize}
	\item \etsy{} dispose d'un environnement \gls{LAMP} et d'une application dite monolithique, c'est-à-dire qu'elle est constituée en une seule partie et donc difficile à faire évoluer lors de grosse mise à jour.
	\item Évidemment, \etsy{} utilisait un gestionnaire de version de son code (\gls{git}) mais n'utilisait pas de branches\footnote{Dans un gestionnaire de version, les branches permettent de séparer les fonctionnalités qui sont en cours de développement ainsi que les différentes versions du code}
	\item Du fait de sa croissance rapide, \etsy{} effectuait 2 à 3 mise à jour par semaine, mais ces dernières étaient effectuées avec difficulté et prenaient environ 4h par livraison. 
\end{itemize}

\epigraph{[deploy were done] twice a week, and each deploy took well over four hours.}{\citetitle{etsyInterview} \cite{etsyInterview}}

Au vu de ces informations et du fait de leur croissance, il était nécessaire de faire évoluer les méthodes de travail et de déploiement pour pouvoir suivre la cadence. L'objectif était donc de passer d'un développement lourd, non organisé à un déploiement rapide permettant un retour rapide sur l'état des déploiements ainsi que plusieurs déploiements par jours.

\todo[color=red]{Cas pratique, 5-6 pages}

a beaucoup utilisé l'open source et y a beaucoup contribué en retour (outil post mortem par ex) %https://github.com/etsy/'

\subsection{Difficulté d'avoir la panacée}

\todo[color=red]{Difficulté d'avoir la panacée, 1-2 pages}

- pourquoi n'existe pas une solution unique ? 

- Pourquoi complexité des processus de déploiement ? 

- chaque outil répond à un besoin, mais existe-t-il un outil all-in-one ? (non)

Solutions existantes

Chef, puppet, ansible, jenkins, travis, docker, terraform, heroku, aws, ...

Combinaison d'outils qui répondent à un besoin, pattern.


\textbf{ATTENTION : NE PAS REPONDRE A LA QUESTION, SEULEMENT SOULEVER DES INTERROGATIONS POUR LE LECTEUR}

Cas pratique (démo entreprise migration) / Comment le généraliser (NE PAS RÉPONDRE A CETTE QUESTION). Cas précis et généraliser, parler des autres solutions. Difficulté d'avoir une solution clé en main, une solution à tout, la panacée

Qu'est ce qui a poussé à l'automatiser ? Complexification de ces étapes (npm) pour pousser à la nécessité l'automatisation. Le fait que l'utilisateur veut toujours plus de simplicité pousse à toujours plus de complexité de réalisation. Permet de se concentrer sur des tâches à forte valeur ajoutée.

Innovation technologique 