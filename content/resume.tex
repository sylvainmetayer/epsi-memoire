\begin{abstract}

{\em \large
Ce document est réalisé à l'aide de \LaTeX. La dernière version de ce document, ainsi que ses sources, sont disponible à l'adresse suivante.

\begin{center}
	\url{https://memoire.epsi.sylvainmetayer.fr}
\end{center}

Pour information, tout mot suivi du symbole \frquote{*} indique une référence vers le glossaire présent page \pageref{glossaire}. Les numéros entre crochets \frquote{[1]} indique une référence vers un élément de la bibliographie disponible page \pageref{bibliographies}
}

\hrulefill

\begin{large}

L'objectif de ce document est de vous montrer les avantages et les moyens de mettre en place une démarche \devops{} et d'automatisation au sein de projet informatique. Ce document, bien que contenant une partie technique, s'adresse à tout public souhaitant le lire.

Ce document va débuter par un historique de l'automatisation, afin de mettre en place le contexte de ce document, puis détaillera les différentes parties qui composent le cycle de vie d'une application. Cela permettra de détailler les différentes étapes et leur utilité. Le lecteur disposera alors d'une vue d'ensemble sur le cycle de vie d'une application.

Nous aborderons ensuite, au travers d'un cas pratique, l'intérêt de la mise en place d'une démarche d'automatisation avec l'entreprise \etsy{}. Cela mettra en évidence les moyens utilisés par l'entreprise pour effectuer sa transition dans une démarche \devops{} et mettre en place des méthodes d'automatisation. Nous verrons également les différentes difficultés à mettre en place un tel système et pourquoi les déploiements/développements se sont complexifiés avec le temps.

Finalement, nous aborderons la partie concrète du sujet avec la façon de mettre en place une démarche d'automatisation et ce, depuis différents points de vue : organisationnel, financier, technique ainsi que les pré-requis à la mise en place d'une telle démarche.

\end{large}
\end{abstract}
