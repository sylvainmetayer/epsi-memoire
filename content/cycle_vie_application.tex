\todo{Cycle de vie d'une application, 8 pages}

Une application, qu'il s'agisse d'une application web ou de bureau se décompose la plupart du temps en plusieurs étapes. Rare, pour ne pas dire quasi-inexistant, sont les projets réussis qui ont été développé sans jalons, sans documentation (bien rédigée ou non), et sans support (technique, utilisateur).

Une application nait du fait de répondre à un besoin. Par exemple, Blablacar\footnote{\url{https://blablacar.fr/}} est né du besoin de répondre à la demande de covoiturage pour se déplacer plus facilement et à moindre coût.

De plus, selon les différentes phases du projet, plusieurs personnes vont être amenées à travailler sur ce dernier. Il est donc primordial de pouvoir établir un suivi de l'application, car si plusieurs personnes interviennent sur un projet, il est tout à fait possible que des personnes aillent et viennent, nécessitant une bonne organisation pour éviter de tout reprendre de zéro à chaque fois.

On peut donc distinguer plusieurs étapes dans le cycle de vie d'une application.

Le but de cette partie est donc de présenter, étape après étape, ces phases.

\todo{Ca serait bien que l'intro de la partie 1 fasse une page au lieu de à peine une demie-page}

\subsection{Expression du besoin}

Tout d'abord, avant de développer une application, il s'agit d'exprimer le besoin auquel elle doit répondre. Il s'agit de la première étape, \emph{l'expression du besoin}. Le but de cette étape est de déterminer si le besoin de l'application est réel, à qui est destinée l'application, et de constituer une première équipe qui portera ce projet.

\subsubsection{Répondre à un besoin}

Ainsi, on va d'abord commencer par savoir à quel besoin répondre. Pourquoi cette idée pourrait-elle fonctionner ? Qu'est ce qui fait que cela peut être rentable, ou même utile ? 

Toute ces questions, il est nécessaire de se les poser lorsque l'on commence à réfléchir à une idée d'application. L'une des premières étapes, généralement, est de vérifier si une application similaire, ou répondant déjà à ce besoin n'existe pas déjà. Dans le cas ou une \todo{Continuer cette partie de répondre à un besoin}

- A quoi va servir le produit, à quel besoin répond-il ?

- Etude de faisabilité / concurrence

- Définition des cibles, early adopters.

- Définition des responsables du projet

- Collecte d'information sur la faisabilité / besoin du projet.

- Exemple : location de trottinettes => besoin / études /

%% https://medium.com/@faureguillaume/etude-du-march%C3%A9-des-trottinettes-%C3%A9lectriques-en-libre-service-en-france-et-dans-le-reste-du-monde-36b3370b3f4c avec image de la micromobilité 

\todo{Expression du besoin}

\subsection{Planification}

- Définition d'un budget

- Négociation commerciale

- Définition risques juridiques / légaux (GDPR, CNIL...).

- Diagramme de Gant pour organisation projet, avec les grandes deadlines.

- Expression du besoin, rédaction cahier des charges

Cahier des charges :permet de définir charte graphique \& architecture projet. Définit objectif. Poser au clair les besoins et spécificités des demandes. Priorisation. Définit qui a quel rôle? Contient des garanties pour le client (SLA...) \& l'entreprise (fonctionnalités hors scope \& co)

- Définition des priorités : Must, Should, May en vue de réaliser un MVP

\todo{Planification}

\subsection{Réalisation}

- Réalisation graphique

- Plusieurs itérations

- Retour client, demande de précisions sur les fonctionnalités

- Développement de tests

\todo{Réalisation}

\subsection{Mise en recette}

- Déploiement en interne de l'application

- Permet de découvrir les premiers bug avec un environnement de production

- Retour sur ce qui était pensé par le client et ce qu'il veut au final (ce bouton, il le veut vraiment la le client en fait ? Ca fait pas très pratique ici..)

- Environnement pré-production / intégration \& co avant mise en production.

\todo{Mise en recette}

\subsection{Qualification}

- Qualification : le client vérifie la conformité de l’application développée aux spécifications établies.

- Premier contact de l'application avec le client

- Evolution potentiellement lourde ("oh, au final je voulais plutôt ça, ça se fait bien ? Alors qu'il faut 10j pour le mettre en place")

- Potentiellement plusieurs déploiement avant validation finale

\subsection{Mise en production}

- Choix de la date important (ex: Transport, pas 10j avant la rentrée et tous les parents qui renouvelent l'abonnement de leur enfant.)

- Prévoir les imprévus : validation application mobile, propagation dns...

- Formation utilisateur

- Promotion client du produit.

- Premier déploiement souvent "stressant"

- Besoin de métrique pour pouvoir déterminer le succès de l'application

\todo{Mise en production}

\subsection{Maintenance}

- Correction de bug au fur et à mesure

- Besoin de métrique pour pouvoir déterminer le statut de l'application, ses performances \& eventuels défauts.

- Suivi des logs 

- Système de remontée d'erreur organisée

\todo{Maintenance}

\subsection{Évolution}

Un projet qui n'évolue pas est un projet mort. Ainsi, il est nécessaire pour une application de se remettre continuellement en question, de chercher et d'implémenter de nouvelles idées, de faciliter la vie de l'utilisateur, de proposer de nouvelles fonctionnalités, de se mettre à jour vis à vis des dernières évolutions technologiques.

\todo{Evolution}

- Vie du projet

- Audit de sécurité, pour se tenir à jour

- Audit de performance

\subsection{Fin de vie d'un projet / Décommissionnement}

\todo{Fin de vie d'un projet}

- Remplacement par une autre solution : transfert des données

- Débat légal sur la conversation ou non des données

- Offrir une solution alternative / une possibilité de récupérer ses données pour les utilisateurs.

\missingfigure{Schéma global reprenant les différents étapes du cycle de vie d'une application}