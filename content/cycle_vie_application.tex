\todo{Cycle de vie d'une application, 8 pages}


Une sous partie par cycle de vie de l'application (démarrage, vente, avant-vente, maintenance...)

Lister étapes à faire, avant vente, appel d'offre, commerciale, gestion exigence et ce qui est développé, backlog lié au autres, livraison, qualité, exigences, organisation...


\subsection{Expression du besoin}


- A quoi va servir le produit, à quel besoin répond-il ?

- Etude de faisabilité / concurrence

- Définition des cibles, early adopters.

- Définition des responsables du projet

- Collecte d'information sur la faisabilité / besoin du projet.

\subsection{Planification}

- Définition d'un budget

- Négociation commerciale

- Définition risques juridiques / légaux (GDPR, CNIL...).

- Diagramme de Gant pour organisation projet, avec les grandes deadlines.

- Expression du besoin, rédaction cahier des charges

Cahier des charges :permet de définir charte graphique \& architecture projet. Définit objectif. Poser au clair les besoins et spécificités des demandes. Priorisation. Définit qui a quel rôle? Contient des garanties pour le client (SLA...) \& l'entreprise (fonctionnalités hors scope \& co)

- Définition des priorités : Must, Should, May en vue de réaliser un MVP

\subsection{Réalisation}

- Réalisation graphique

- Plusieurs itérations

- Retour client, demande de précisions sur les fonctionnalités

- Développement de tests

\subsection{Mise en recette}

- Premier contact de l'application avec le client

- Retour sur ce qui était pensé par le client et ce qu'il veut au final

- Evolution potentiellement lourde ("oh, au final je voulais plutôt ça, ça se fait bien ? Alors qu'il faut 10j pour le mettre en place")

- Potentiellement plusieurs déploiement avant validation finale

- Environnement pré-production / intégration \& co avant mise en production.

\subsection{Mise en production}

- Choix de la date important (ex: Transport, pas 10j avant la rentrée et tous les parents qui renouvelent l'abonnement de leur enfant.)

- Prévoir les imprévus : validation application mobile, propagation dns...

- Promotion client du produit.

- Premier déploiement souvent "stressant"

- Besoin de métrique pour pouvoir déterminer le succès de l'application

\subsection{Maintenance}

- Correction de bug au fur et à mesure

- Besoin de métrique pour pouvoir déterminer le statut de l'application, ses performances \& eventuels défauts.

- Suivi des logs 

- Système de remontée d'erreur organisée

\subsection{Évolution}

- Vie du projet

- Audit de sécurité, pour se tenir à jour

- Audit de performance

\subsection{Fin de vie d'un projet / Décommissionnement}

- Remplacement par une autre solution : transfert des données

- Débat légal sur la conversation ou non des données

- Offrir une solution alternative / une possibilité de récupérer ses données pour les utilisateurs.

