Une application, qu'il s'agisse d'une application web ou de bureau, ne se réalise pas du jour au lendemain d'un claquement de doigt. Sa mise en place et son déroulement se décomposent la plupart du temps en plusieurs étapes. Rares, pour ne pas dire quasi-inexistant, sont les projets réussis qui ont été développés sans jalons, sans documentation (bien rédigée ou non) et sans support (technique ou utilisateur). C'est pour répondre à un (ou plusieurs) besoins que l'on vient à développer une application. Par exemple, Blablacar\footnote{\url{https://blablacar.fr/}} est né du besoin de répondre à la demande de covoiturage pour se déplacer plus facilement (\emph{recherche facile de trajet, communication entre les membres...}) et à moindre coût.

Selon les différentes phases du projet, plusieurs personnes vont être amenées à travailler sur ce dernier. Il est donc primordial de pouvoir établir un suivi de l'application. Si plusieurs personnes interviennent sur un projet, il est tout à fait possible que des personnes aillent et viennent, nécessitant une bonne organisation pour éviter de tout reprendre de zéro à chaque fois. On peut donc distinguer plusieurs étapes dans le cycle de vie d'une application.

On peut ainsi représenter chacune de ses étapes avec le schéma présent sur la figure \ref{fig:cycleVieProjet}. Toutes les étapes du cycle de vie d'une application ne sont pas présentes (notamment les parties financières par exemple) et certains projets n'ont peut-être pas besoin de chacune de ses étapes. Ce schéma se veut général, afin de donner une vision d'ensemble de la façon dont se déroule un projet.

Le but de cette partie est de présenter, les une après les autres, les étapes présentes sur le schéma \ref{fig:cycleVieProjet}. Ce schéma est également disponible pour une meilleure lisibilité en annexe \ref{annexe:cycleVieProjet}.

\newImage[H]{0.30}{cycle-vie-projet.png}{Schéma simplifié du cycle de vie d'un projet}{cycleVieProjet}

\subsection{Répondre à un besoin}

La première étape du développement d'une application est \emph{l'expression du besoin} durant laquelle il s'agit de détailler et définir le besoin auquel le produit doit répondre. Le but de cette étape est de déterminer si le besoin de l'application est réel, qui est amené à utiliser cette application et de constituer une première équipe qui portera ce projet.

Ainsi, on va d'abord commencer par savoir à quel besoin répondre. Pourquoi cette idée pourrait-elle fonctionner ? Qu'est-ce qui fait que cela peut être rentable, ou même utile ? Il est nécessaire de se poser toutes ces questions lorsque l'on commence à réfléchir à une idée d'application.

L'une des premières étapes, peut être de vérifier si une application similaire, ou répondant déjà à ce besoin n'existe pas déjà. Dans le cas ou une application similaire existe déjà, on peut tenter de se démarquer, en proposant des fonctionnalités supplémentaires, une interface améliorée... Ou parfois tout simplement en utilisant le produit existant et en passant à une autre idée.\footnote{Chaque idée ne débouche pas forcément sur un produit fini, il faut souvent persévérer pour trouver une bonne idée et se démarquer !}

Il s'agit ensuite de déterminer son but, le besoin auquel le produit répond. Par exemple, les \gls{vtc} répondent à un besoin de déplacement instantané d'un point A à un point B et une facilité de réservation, peu importe l'endroit ou la personne se trouve. De plus, il faut également se renseigner sur les utilisateurs potentiels avant de lancer son produit. Si des services similaires existent, il est nécessaire de pouvoir déterminer une tranche d'utilisateurs à viser. Si le service est nouveau, il faut pouvoir déterminer les \gls{earlyadopter} du service. Cela peut se faire au travers de démonstration d'une version bêta du site, par un sondage en ligne afin de déterminer les cibles les plus à même d'utiliser le service... Pour bien déterminer ses \gls{earlyadopter}, on peut se poser les questions suivantes afin de déterminer des \Glsplural{persona} :

\begin{itemize}
	\setlength\itemsep{-0.5em}
	\item Quel est l'âge de la personne ciblée ?
	\item Quel est le métier et la catégorie socio-professionnelle de la personne ciblée ?
	\item Quel est le sexe de la personne ciblée ? 
	\item Où habite la personne ciblée ? En ville, en campagne... ? 
	\item Pourquoi a-t-il besoin de ce service ? 
	\item Qu'est-ce qui peut l'empêcher d'utiliser le service (aspect financier, lenteur du service...) 
	\item Quel serait un (ou plusieurs) cas d'usage de la personne ciblée avec le service ?
\end{itemize}

Une fois la cible établie, il convient également de savoir si la solution est viable financièrement. En effet, on peut avoir la meilleure idée du monde, mais si on ne peut gérer la partie financière du projet, ce dernier est voué à l'échec. Cette étude financière va inclure une étude sur le temps de développement d'une première version, les coûts de communication, des serveurs, des déplacements (forums, congrès...) pour présenter l'application\ldots

De l'autre côté, on va déterminer les recettes de l'application, en estimant les revenus que pourrait générer les \gls{earlyadopter} et les autres utilisateurs, ainsi que les autres sources de revenus potentiels (publicité, crowfounding...). Ainsi, lorsque l'on va contacter les \gls{earlyadopter} pour s'informer de leur intérêt sur le projet via, par exemple, un sondage en ligne, il n'est pas rare de voir une partie concernant les tarifs que les utilisateurs seraient prêts à dépenser, avec des questions telle que :

\begin{itemize}
	\setlength\itemsep{-0.5em}
	\item Quel est le prix minimum sous lequel le service est considéré de mauvaise qualité?
	\item Quel est le prix maximum acceptable pour l'utilisation de ce service?
\end{itemize}

Ces questions permettent de définir de potentiels avertissements sur la viabilité financière du projet. Si l'on constate que peu d'utilisateurs sont prêt à investir de l'argent pour utiliser le projet, il faudra se demander si l'ajout d'une version gratuite et d'une version premium payante, ne serait pas plus intéressante. On peut également penser à se tourner vers d'autres entreprises prête à financer le projet en échange de publicité ou d'exclusivité par exemple.

%% https://medium.com/@faureguillaume/etude-du-march%C3%A9-des-trottinettes-%C3%A9lectriques-en-libre-service-en-france-et-dans-le-reste-du-monde-36b3370b3f4c avec image de la micromobilité 

\subsection{Expression \& Planification du besoin}

Une autre étape intervenant en amont du projet est \emph{l'expression du besoin} et sa \emph{planification} . Cette étape est très importante et n'est pas à négliger puisqu'elle va permettre de déterminer des jalons qui définiront les moments clés du cycle de vie du projet. De plus, comme on peut le voir sur la figure \ref{fig:cycleVieProjet} vue précédemment, ces deux étapes sont présentes durant tout le cycle de vie du projet. Bien qu'elles soient séparées sur le schéma \ref{fig:cycleVieProjet}, elles sont assez liées puisqu'il est en général compliqué de planifier quelque chose que l'on ne connait pas.

Commençons par détailler l'expression du besoin. Cela consiste à découper les fonctionnalités de l'application et à en détailler le fonctionnement, dans un document (souvent constitué de plusieurs dizaines de pages) qui sera transmis à l'équipe en charge du développement. L'objectif est de s'assurer qu'il n'y ait pas d'ambiguïté sur les fonctionnalités à réaliser. Il s'agit d'une étape à ne pas négliger puisqu'elle va servir de support à l'équipe de développement lors des phases de réalisation. Ainsi, ce sont les phases d'expression du besoin qui définissent les règles métiers de l'application. 

{\em
Avant de continuer, il est important d'effectuer un point sur ce qu'est une règle métier. 
}

\todo[color=blue]{Définir règle métier}

L'expression du besoin contient aussi des garanties tant pour le client que pour le prestataire. Ces garanties peuvent être des assurances de stabilité du service (\gls{SLA}), une charte graphique imposée ou au contraire avec une certaine liberté laissée au prestataire, les responsables et rôles du projet, côté client (personne référente du projet, à contacter en cas de questions) et prestataire (chef de projet par exemple).

C'est également durant cette phase que l'on va décrire les interactions de l'application avec d'autre système. Par exemple, prenons une application de covoiturage. Cette dernière va certainement interagir avec plusieurs autres systèmes.

\begin{itemize}
	\setlength\itemsep{0em}
	\item L'application a besoin d'enregistrer les réservations des utilisateurs. Pour cela, elle va interagir avec une base de données afin de stocker les réservations.
	\item L'application aura besoin de calculer les trajets les plus optimisés pour pouvoir proposer un trajet rapide à ses utilisateurs. Elle va donc interagir avec des systèmes de cartographies externes, tel qu'OpenStreetMap\footnote{\url{https://openstreetmap.org/}}.
	\item Il va falloir enregistrer les paiements des utilisateurs à la fin de leur réservations. Pour cela, il va falloir interagir avec un système de paiement. Étant donné la complexité et la sécurité des données bancaires, il s'agira sûrement d'un système externe.
	\item A des fins de statistiques sur le taux d'utilisation, le site utilisera peut-être un système de suivis statistiques tel que Matomo\footnote{\url{https://matomo.org}} 
\end{itemize}

Dans la méthodologie \gls{cycle_v}, l'expression du besoin était réalisé par le client sans forcément communiquer avec l'équipe de développement qui effectuait alors ses propres interprétations si le document n'était pas assez précis, ce qui pouvait donner lieu à des erreurs d'interprétations et provoquer une différence entre ce que le client s'attendait à recevoir et ce qu'il obtenait réellement à la fin. 

\newImage{0.5}{gestionProjet.jpg}{Communication au sein d'un projet, allégorie - \url{https://www.anyideas.net}}{gestionProjet}

Comme le montre l'image \ref{fig:gestionProjet}, souvent utilisée pour illustrer de manière humoristique les dérives potentielles de la gestion de projet, plus le nombre de personnes est important sur un projet et plus la déformation de l'information (\emph{au travers des différents intermédiaires}) est présente, plus les différences d'interprétation peuvent être importantes. 

Une méthode de travail alternative a commencé à s'imposer depuis une dizaine d'années et a pour objectif de combler ce manque de communication et de faire des retours rapides au client. Dans cette méthodologie, dites \emph{méthodologie agile}, la phase de réalisation est caractérisée sous forme de sprint, qui inclus non pas toutes les fonctionnalités demandées, mais seulement une partie, qui peut-être réalisable dans un temps convenu avec le client. Le but est de développer fonctionnalités par fonctionnalités, et de pouvoir présenter quelques choses de concret au client lors des fins de sprint. Ainsi, à chaque itération, on se rapproche un peu plus de la version complète du produit, et on apporte constamment de la valeur ajoutée au produit. Le client dispose ainsi d'une version \emph{incomplète, mais utilisable} à chaque fin de sprint et c'est au fur et à mesure des itérations que son produit deviendra de plus en plus complet. Si, pour une raison ou une autre, le client vient à manquer de financement pour continuer le projet, il peut en théorie toujours utiliser la dernière version disponible, même si elle ne dispose pas de toutes les fonctionnalités souhaitées, en attendant de disposer de fonds supplémentaires pour pouvoir continuer les développements.

Il est ainsi nécessaire de trier les fonctionnalités par priorité. En effet, selon les délais impartis, il ne sera peut-être pas possible de voir toutes les fonctionnalités présentes dans la première version. Cela impliquerait un \gls{timetomarket} plus élevé, et donc des risques de voir d'autres concurrents se positionner sur le marché. Il est donc plus prudent de définir les fonctionnalités clés de l'application afin d'avoir un produit à forte valeur ajoutée et de prévoir les autres fonctionnalités dans les évolutions futures. On peut prioriser les usages d'une application en plusieurs catégories, détaillées ci-dessous.

\begin{description}
	\item [Les \emph{must-have}]. Ce sont les fonctionnalités sans lesquelles l'application n'est pas utilisable. Par exemple, si le service propose des achats, mais que le tunnel d'achat ne gère pas les paiements, il sera impossible de compléter la moindre commande, et par conséquent, le site ne sera pas utilisable. On peut ainsi dire que ces fonctionnalités correspondent au \gls{mvp} du projet.
	\item [Les \emph{should have}]. Ce sont des fonctionnalités essentielles, mais qui ne sont pas bloquantes pour l'utilisation de l'application. Par exemple, dans le cas d'une connexion à un service, il peut être intéressant de proposer un bouton permettant de garder la session active, pour éviter d'avoir à se reconnecter. Néanmoins, si cela n'est pas présent, on peut tout de même se connecter.
	\item [Les \emph{could have}] Ce sont des fonctionnalités dites de confort. Elles ne sont pas prioritaires, mais il serait intéressant de les développer puisqu'elles apportent un certain confort d'utilisation. Par exemple, proposer un aperçu automatique sur un éditeur \gls{markdown} est une fonctionnalité qui apporte un certain confort à l'utilisateur, puisqu'il a un retour immédiat. Cela ne change pas le fonctionnement de l'éditeur \gls{markdown} pour autant.
	\item [Les \emph{won't have}] Ce sont des fonctionnalités intéressantes mais qui ne sont pas prévues pour le moment. Elles pourront néanmoins être étudiées pour pouvoir être implémentées plus tard. Cela peut par exemple être un système de connexion via des sites externes (tel que par exemple Facebook ou Twitter).
\end{description}

%% https://fr.wikipedia.org/wiki/M%C3%A9thode_MoSCoW

Concernant la planification d'un projet, l'organisation de ce dernier passe souvent par l'élaboration de \glsplural{gantt} (ou équivalent) pour représenter de façon graphique les jalons du cycle de vie d'une application (fin des développements, phase de qualification, mise en production, réunions périodiques de suivi, durée de chaque sprint\ldots).

\newImage{0.30}{gantt.png}{Exemple de Gantt - \url{wikimedia.org}}{gantt}

Un \gls{gantt} contient donc les différents jalons du cycle de vie d'une application, mais il est aussi nécessaire de prévoir les éléments pouvant être bloquants. Par exemple, dans le cas de collecte de données utilisateurs afin de réaliser un traitement, il convient de prévoir un délai pour effectuer les démarches auprès de la \gls{cnil}, prévoir un délai de validation du contrat avec les différents prestataires, les délais de réponse au sein de l'organisation afin de valider le projet, les démarches administratives (conditions générales d'utilisations à rédiger par des cabinets d'avocats)... De plus, on peut également prévoir les dates des \gls{kickoff} et autres ateliers aidant à exprimer les besoins sur certaines fonctionnalités, en collaboration avec le client. Selon la granularité souhaitée, on pourra également y faire figurer, à titre indicatif pour les équipes en charge de la réalisation, les dates à partir de laquelle le client va faire la promotion de son application (\emph{et par conséquent, la date à laquelle l'application doit être plus ou moins prête})

\subsection{Réalisation}

La réalisation de l'application intervient après la validation du besoin avec le client. Démarre alors la conception de maquette basée sur les exigences de la charte graphique, la réflexion sur la façon d'implémenter les différentes fonctionnalités demandées, ainsi que les réalisations à proprement parler pour concevoir le site. Ces étapes peuvent s'étaler, dans le cas d'une méthodologie agile, sur plusieurs sprints selon la complexité du projet.

Une phase de réalisation, en particulier au commencement d'un projet, n'est pas l'équipe de développement tapant à toute vitesse sur son clavier pour livrer au plus vite l'application. L'une des premières étapes est de définir un environnement de développement, souvent avec plusieurs outils additionnels, afin d'aider la phase de développement.

\todo[color=orange]{Ne pas citer d'exemple d'outils additionnels, cela sera fait dans la partie 3, phase locale de développement.}

Il s'agit ensuite de déterminer la façon dont vont être réalisés les développements. Si le projet est un blog basique, on peut envisager l'utilisation d'un \gls{cms}, puisqu'il n'est peut-être pas utile de réinventer la roue pour ce besoin, sachant que des dizaines de solutions existent déjà. Il faudra alors réaliser une étude afin de déterminer l'outil le plus adapté parmi ceux existants, tout en prenant en compte les contraintes du projet (\emph{par exemple, si le client souhaite exclusivement un projet en \gls{PHP}, il ne sert à rien de lui proposer un \gls{cms} en java}).

Si les besoins sont plus complexes, on peut envisager deux possibilités.

\begin{enumerate}
	\item L'utilisation de \gls{framework} qui permet de démarrer rapidement un projet, à condition de maitriser le \gls{framework}. Ce dernier peut en effet nécessiter une courbe d'apprentissage plus ou moins importante selon ce que l'on souhaite en faire. Le \gls{framework} est une boite à outil permettant de définir des normes de travail afin de gagner en efficacité et de fournir un socle de travail commun à toute l'équipe de développement mais il peut ne pas être adapté à tous les cas, au risque de rapidement se retrouver dans une position ou l'on va constamment \emph{tordre} le comportement du \gls{framework} pour obtenir le résultat escompté.
	\item Un développement \emph{from scratch} également appelé développement maison qui consiste à construire une solution depuis zéro, ce qui implique un temps de développement potentiellement plus long et une maintenance nécessitant de se former sur le projet spécifiquement. En effet, les \gls{framework} permettent une intervention d'un développeur externe au projet plus facile, à condition qu'il ai connaissance du \gls{framework}, tandis que dans le cas d'un projet \emph{from scratch}, il est plus compliqué d'intervenir rapidement sans connaitre le projet. Néanmoins, cela offre une totale liberté et ne ferme pas de porte, comme peuvent le faire certains \gls{framework} en déconseillant ou forçant (\emph{à tort ou à raison}) certaines pratiques.
\end{enumerate}

Une des phases souvent négligées dans les projets est la phase de tests. En effet, il est essentiel de garantir le bon fonctionnement du projet, autrement que par un \frquote{Cela marche sur mon poste, c'est bon !}. Le client souhaite des tests afin de s'assurer du bon fonctionnement de son application, et pour être rassuré\footnote{Même s'il est impossible de garantir une application sans bug} quant à l'apparition d'un bug en production, pouvant potentiellement impacter son activité, chiffre d'affaires ou la relation avec ses clients. Il existe différents types de test, chacun ayant une responsabilité différentes. Nous aborderons les détails à propos des tests dans la partie sur \frquote{\nameref{importance-test}} (p. \pageref{importance-test})

C'est également lors de la phase de réalisation que l'on va porter attention plusieurs éléments. On peut notamment citer les éléments suivants qui sont à prendre en compte.

\begin{description}
	\item [La sécurité et le respect de la vie privée] des utilisateurs afin de garantir que les données de ces derniers soient en sûreté. 
	
	On néglige trop souvent cette partie, par manque de temps, de moyens engagés ou par méconnaissance des risques et on se retrouve, même parmi les grandes multinationales avec des scandales médiatiques qui peuvent entrainer de lourdes amendes, une perte de confiance de la part des utilisateurs et perte de chiffres d'affaires.\footnote{\url{https://dayssincelastfacebookscandal.com/}}
	
	Heureusement, avec la mise en place du \gls{RGPD} (qui oblige à limiter les traitements de données personnelles uniquement lorsque cela est nécessaire) en mai 2018, on peut espérer de meilleure prise en compte de ces contraintes par les entreprises dans le futur.
	\item [L'accessibilité de l'application] qui peut être une clause obligatoire selon le type d'application et le client.
	
	La norme \gls{RGAA}, par exemple, impose des règles strictes pour les sites pour lesquels l'accessibilité est primordiale (par exemple les sites institutionnels) afin que chacun puisse accéder au contenu.
	\item [Le référencement] car un site qui n'est pas référencé est un site qui ne sera pas visible. Le référencement est à prendre en compte dès le début des phases de développement, via l'application des bonnes pratiques de référencement.
	\item [Le green--IT] qui est un mouvement visant à réduire la consommation énergétique des applications, au travers de bonnes pratique. Cela peut aller d'une gestion de cache amélioré à de la configuration spécifique sur les serveurs pour permettre de meilleures performances tout en passant par l'optimisation du code écrit par les développeurs.
	\item [Les bonnes pratiques en général] car une application qui respectent des bonnes pratiques, défini par un organisme reconnu et approuvé par une grande communauté, est une application qui sera plus simple à maintenir et qui entrainera potentiellement moins de bugs\footnote{Et il est toujours plus agréable de se dire que l'on a réalisé quelque chose correctement plutôt que livrer un travail que l'on sait de mauvaise qualité}. Pour les développements de site web, un ensemble de bonnes pratiques existe, il s'agit de la \citetitle{opquast-best-practices} \cite{opquast-best-practices}, qui propose une certification\footnote{\url{https://opquast.com/fr/}} reconnue professionnellement aux personnes souhaitant la passer afin de valider leur expertise dans la mise en place des bonnes pratiques web.
\end{description}

\subsection{Qualification}

\todo[color=orange]{Toutes les infos sont présentes dans cette sous partie, par contre ça me semble être le bordel niveau organisation des paragraphes. Besoin de repasser dessus pour rendre la lecture plus claire.}

Comme il a été dit précédemment, une phase de réalisation ne s'effectue pas en une fois\footnote{sauf exceptions, par exemple dans le cas d'un projet très court}, et va donc être découpée sous forme de sprint. A la fin de chaque sprint, dont la durée est déterminée durant la phase de planification, il va être présenter au client le travail réalisé. Comme indiqué sur le schéma \ref{fig:cycleVieProjet} (p. \pageref{fig:cycleVieProjet}), on peut voir cela comme une succession de \emph{mini-phase} de réalisation et de qualification. Cela va permettre d'obtenir un retour du client avec des éventuelles précisions sur le besoin, des demandes de modifications\ldots Ces itérations vont alors se répéter jusqu'à obtenir une validation du client qui permettra alors la mise en production.

La phase de qualification permet ainsi au client d'avoir un contact direct avec son application. C'est également durant ce temps que sont lancés plusieurs tests de performance, des tests de qualités... Pour s'assurer que l'application se comporte correctement.

Néanmoins, la phase de qualification débute rarement par la présentation de l'application au client. En effet, avant cela, l'application va d'abord être déployée en interne, sur un environnement qui sera similaire à celui de la future production. Cela va permettre de vérifier la validité de la configuration de production. De plus, c'est à ce moment là que l'on va tester la procédure de déploiement de l'application et l'ajuster afin de s'assurer qu'elle permettra un déploiement sans accroche, cette fois sur un environnement de production.

La phase de qualification nécessite beaucoup de rigueur. Plus cette phase est effectuée avec sérieux et rigueur, moins il y a de chance de retrouver un bug en production. Selon les erreurs/retours remontés par le client, elles seront alors évaluées en fonction de l'expression du besoin initial, et catégorisée en bug (et donc corrigée) ou alors en évolution, qui ne sera alors pas traitée tout de suite et entrainera des coûts supplémentaires. Cela peut être le cas si le client, qui souhaitait initialement un formulaire de connexion simple à un service, découvre qu'il ne peut se connecter via un \gls{LDAP}. Étant donné que le client n'avait pas exprimé son besoin, il est logique que ce dernier ait mal été interprété. \emph{Heureusement, ces désagréments sont censés être évités si l'expression du besoin, réalisée en collaboration entre le client et son prestataire, est bien réalisée et ne laisse pas place à des zones d'ombres.}

De plus, selon le projet, l'application finale pourra être déployée sur un serveur appartenant au prestataire ou bien chez un prestataire externe qui effectuera uniquement la partie hébergement. Si c'est le cas, il s'agit alors de tester que la communication entre les deux prestataires est bonne, en fournissant la procédure d'installation et en testant sur un environnement similaire à la production le bon fonctionnement de l'application.

\subsection{Mise en production}

% \newImage[H]{0.25}{no-deploy-friday.jpg}{\frquote{\emph{Never deploy on friday}} - \url{http://commitstrip.com/}}{no-deploy-friday}

La mise en production correspond au déploiement de l'application pour les utilisateurs finaux. Il s'agit d'une étape importante, qui marque l'aboutissement d'une période plus ou moins longue de travail.

Avant de passer à la mise en production, il est important de vérifier plusieurs éléments. Ci-dessous se trouve quelques exemples qui peuvent être bloquant pour la mise en production d'une application.

\textit{La date de mise en production est-elle cohérente ?} Par exemple, dans le cas d'un site marchand, un déploiement deux semaines avant Noël est peu recommandé puisqu'il s'agit d'une période très importante. S'il s'agit du premier déploiement et donc de la mise en ligne initial du site, il est sûrement trop tard pour pouvoir être référencée par les moteurs de recherche correctement. S'il s'agit d'une mise à jour d'un site déjà existant, le risque d'interrompre le trafic, même pour une mise à jour mineure ou aucun problème n'a été relevé, est important et pourrait provoquer de sérieuse pertes pour le client.

\textit{Si la phase de planification a bien été menée, toutes les démarches, qu'elles soient administratives, techniques ou juridiques, devraient être effectuées avant la mise en production.} Imaginez le lancement d'une application mobile, avec toute la phase de communication qui s'en précède, bloqué à cause d'une attente de validation par le magasin d'application. Cela pourrait donner une mauvaise image du client et montrer une mauvaise anticipation de la part du prestataire, qui n'a pas pensé à prévenir le client de ces potentiels délais. Dans le même registre, on peut également penser à la propagation\footnote{Même si cela tend à être de plus en plus rapide} du \gls{DNS} pour un domaine récemment enregistré ou mis à jour. Si l'application est déployée mais que le domaine n'est pas encore propagé et que par conséquent, les clients ne sont pas en mesure d'accéder au site, cela donne une mauvaise image et risque de surcharger inutilement le support.

Une fois ces étapes validées, la mise en production peut alors commencer. Cette dernière est effectuée par le prestataire, qui va communiquer avec le client les dates de début et de fin estimée d'intervention, ainsi que le statut de la mise en production. Si cette dernière a été réalisée avec succès, il va falloir surveiller son activité afin de s'assurer que l'application se comporte bien une fois déployée. Si la mise en production s'est mal déroulée, il va falloir investiguer sur la cause de l'échec et, si l'application a déjà été déployée auparavant, effectuer un retour en arrière afin de ne pas laisser l'application indisponible trop longtemps. 

Une fois la mise en production effectuée avec succès, et dans le cas d'un outil interne, le client (ou le prestataire, selon les clauses du contrat) va devoir former les utilisateurs finaux afin de s'assurer que tous soient en mesure de pouvoir utiliser l'application. Dans le cas d'un site à destination du public, il va falloir communiquer pour indiquer que le site est disponible et éventuellement rédiger des articles pour expliquer l'utilisation du service\footnote{Mais cela est totalement optionnel, car tous les sites sont utilisable de façon accessible et intuitive, c'est bien connu!}

Une fois la mise en production effectuée, une batterie de test est en général effectuée, afin de s'assurer que l'application est bien déployée correctement et répond de façon voulue. Une fois cela fait, il va falloir s'assurer de pouvoir monitorer le statut de l'application afin d'être en mesure de réagir rapidement et avec toutes les données permettant la résolution de l'incident.

\subsection{Maintenance}

La phase de maintenance peut être très calme comme très stressante, selon le sérieux accordé lors des phases de réalisation et de qualification. Elle intervient suite à la mise en production, et tout au long du cycle de vie du projet. Ses clauses sont définies lors de l'établissement du contrat entre le client et le prestataire. Il convient d'y prêter attention puisqu'il est possible d'avoir une maintenance très poussée, ou une très sommaire, selon les contenus des clauses du contrat. Évidemment, plus la maintenance est de qualité, plus le coût de cette dernière augmentera.

La maintenance inclut donc le suivi de l'application, via des métriques afin de déterminer le statut de l'application, ses performances et éventuels défauts. Si une anomalie est relevée par le client ou des utilisateurs, elle sera alors remontée aux équipes de développement qui prendront en charge l'anomalie et proposeront un correctif\ldots

Il est important de ne pas négliger le monitoring de l'application, sous peine de ne pas détecter à temps des potentiels bugs. Par exemple, si l'on constate que le site répond correctement mais parfois lentement et que la charge du serveur qui héberge ce site est constamment haute, il faudra alors investiguer pour déterminer si cela provient de l'application, de l'environnement dans lequel elle s'exécute (\emph{exemple : le serveur héberge plusieurs applications, et c'est une autre application qui occupe toutes les ressources}) ou une cause externe. Les logs de l'application sont alors très utile pour pouvoir déterminer les causes de ces anomalies.

\subsection{Évolution}

Un projet qui n'évolue pas est un projet mort. Ainsi, il est nécessaire pour les mainteneurs du projet de se remettre continuellement en question, de chercher et d'implémenter de nouvelles idées, de faciliter la vie de l'utilisateur, de proposer de nouvelles fonctionnalités, de se mettre à jour vis à vis des dernières évolutions technologiques.

\todo[color=cyan]{Evolution}

- Vie du projet

- Audit de sécurité, pour se tenir à jour

- Audit de performance

\subsection{Fin de vie d'un projet / Décommissionnement}

\todo[color=cyan]{Fin de vie}

Éventuellement, un projet attendra sa fin de vie. Cela peut provenir de plusieurs raisons.

\begin{itemize}
	\setlength\itemsep{-0.5em}
	\item Un nouveau concurrent a gagné des parts de marché, entrainant une baisse de l'utilisation et de la rentabilité du service.
	\item Suite à une erreur de communication, ou une faille de sécurité, le projet est déserté par ses utilisateurs.
	\item Le projet est repris par un autre prestataire pour une refonte, ou tout simplement par choix du client (choix économique, mauvaise entente avec le prestataire actuel...).
	\item Le produit ne répond plus à un besoin (\emph{Par exemple, le produit a été réalisé pour un évènement ponctuel et n'a donc plus d'utilité une fois ce dernier passé})
\end{itemize}

Quoiqu'il en soit, il faut pouvoir gérer la fin de vie d'un projet correctement afin de repartir sur d'autres projets sereinement. 

Si l'application est reprise par un autre prestataire, il va se passer une période dite de réversibilité. Durant cette période, le prestataire actuel sera alors en communication afin de fournir les différents livrables pour assurer une bonne transition. 

Si l'application est en fin de vie, il faudra alors communiquer avec les utilisateurs pour leur proposer de migrer vers des solutions alternatives et leur proposer un moyen de récupérer leur données.
