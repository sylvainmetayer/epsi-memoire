\todo{Cycle de vie d'une application, 8 pages}

Une application, qu'il s'agisse d'une application web ou de bureau se décompose la plupart du temps en plusieurs étapes. Rare, pour ne pas dire quasi-inexistant, sont les projets réussis qui ont été développé sans jalons, sans documentation (bien rédigée ou non), et sans support (technique, utilisateur).

Une application nait du fait de répondre à un besoin. Par exemple, Blablacar\footnote{\url{https://blablacar.fr/}} est né du besoin de répondre à la demande de covoiturage pour se déplacer plus facilement et à moindre coût.

De plus, selon les différentes phases du projet, plusieurs personnes vont être amenées à travailler sur ce dernier. Il est donc primordial de pouvoir établir un suivi de l'application, car si plusieurs personnes interviennent sur un projet, il est tout à fait possible que des personnes aillent et viennent, nécessitant une bonne organisation pour éviter de tout reprendre de zéro à chaque fois.

On peut donc distinguer plusieurs étapes dans le cycle de vie d'une application.

Le but de cette partie est donc de présenter, étape après étape, ces phases.

\todo{Ca serait bien que l'intro de la partie 1 fasse une page au lieu de à peine une demie-page}


\subsection{Répondre à un besoin}

Tout d'abord, avant de développer une application, il s'agit d'exprimer le besoin auquel elle doit répondre. Il s'agit de la première étape, \emph{l'expression du besoin}. Le but de cette étape est de déterminer si le besoin de l'application est réel, à qui est destinée l'application, et de constituer une première équipe qui portera ce projet.

Ainsi, on va d'abord commencer par savoir à quel besoin répondre. Pourquoi cette idée pourrait-elle fonctionner ? Qu'est ce qui fait que cela peut être rentable, ou même utile ? Il est nécessaire de se poser toute ces questions lorsque l'on commence à réfléchir à une idée d'application. 

L'une des premières étapes, peut être est de vérifier si une application similaire, ou répondant déjà à ce besoin n'existe pas déjà. Dans le cas ou une application similaire existe déjà, on peut tenter de se démarquer, en proposant des fonctionnalités supplémentaire, une interface améliorée... Ou parfois tout simplement en utilisant le produit existant et en passant à une autre idée.\footnote{Chaque idée ne débouche pas forcément sur un produit fini, il faut souvent persévérer pour trouver une bonne idée et se démarquer !}

Il s'agit ensuite de déterminer son but, le besoin auquel le produit répond. Par exemple, les VTC\footnote{voiture de transport avec chauffeur} répondent à un besoin de déplacement instantané d'un point A à un point B et une facilité de réservation (le plus souvent via une application, avec géolocalisation des VTC les plus proches).  

De plus, il faut également se renseigner sur les utilisateurs potentiels avant de lancer son produit. Si des services similaires existent, il est nécessaire de pouvoir déterminer une tranche d'utilisateurs à viser. Si le service est nouveau, il faut pouvoir déterminer les \gls{earlyadopter} du service. Cela peut se faire au travers de démonstration d'une version bêta du site, par un sondage en ligne afin de déterminer les cibles les plus à même d'utiliser le service... \todo{Je suis sur qu'on a eu un cours / une ressource sur les early adopter ?}. Pour bien déterminer ses \gls{earlyadopter}, on peut se poser les questions suivantes afin de déterminer des \glsplural{persona} : 

\begin{itemize}
	\item Quel est l'age de la personne ciblée ?
	\item Quel est le métier et la catégorie socio-professionnelle de la personne ciblée ?
	\item Quel est le sexe de la personne cibleé ? 
	\item Ou habite la personne cible ? En ville, en campagne... ? 
	\item Pourquoi a-t-il besoin de ce service ? 
	\item Qu'est ce qui peut l'empêcher d'utiliser le service (aspect financier, lenteur du service...) 
	\item Quel serait un (ou plusieurs) cas d'usage de la personne ciblée avec le service?
	\item ...
\end{itemize}

Une fois la cible établie, il convient également de savoir si la solution est viable financièrement. En effet, on peut avoir la meilleure idée du monde, mais si on ne peut gérer la partie financière du projet, ce dernier est voué à l'échec. Cette étude financière va inclure une étude sur le temps de développement d'une première version, les coûts de communication, des serveurs, des déplacements (forums, congrès...) pour présenter l'application... De l'autre côté, on va déterminer les recettes de l'application, en estimant les revenus que pourrait générer les \gls{earlyadopter} et les autres utilisateurs, ainsi que les autres sources de revenus potentiels (publicité, crowfounding...). Ainsi, lors de la définition des \gls{earlyadopter} via un sondage en ligne, il n'est pas rare de voir une partie concernant les tarifs que les utilisateurs seraient prêt à dépenser, avec des questions telle que par exemple : 

\begin{itemize}
	\item Quel est le prix minimum sous lequel le service est considéré de mauvaise qualité?
	\item Quel est le prix maximum acceptable pour l'utilisation de ce service?
\end{itemize}

- Étude de faisabilité / concurrence

- Définition des responsables du projet

- Collecte d'information sur la faisabilité / besoin du projet.

- Exemple : location de trottinettes => besoin / études /

%% https://medium.com/@faureguillaume/etude-du-march%C3%A9-des-trottinettes-%C3%A9lectriques-en-libre-service-en-france-et-dans-le-reste-du-monde-36b3370b3f4c avec image de la micromobilité 

\todo{Expression du besoin}

\subsection{Planification}

Une autre étape intervenant en amont du projet est la \emph{planification}. Cette étape est très importante et n'est pas à négliger puisqu'elle va permettre de déterminer des jalons qui définiront les moments clés du cycle de vie du projet.

\begin{itemize}
	\item Les dates des ateliers de définitions du besoin entre équipe de développement et client
	\item Les dates de recette et de pré-production du service
	\item La validation client
	\item Les différentes réunions de suivi du projet 
	\item Son annonce au public
	\item Sa mise en production
	\item ...
\end{itemize}

La planification est également une étape qui sera présente durant tout le cycle de vie de l'application, jusqu’à sa fin de vie. L'organisation du projet passe souvent par l'élaboration d'un \glsplural{gantt} pour représenter de façon graphique les jalons du cycle de vie d'une application.

\newImage{0.4}{gantt.png}{Exemple de \gls{gantt} - source : wikimedia.org}{gantt}

Un \gls{gantt} contient donc les différents jalons du cycle de vie d'une application, mais il est aussi nécessaire de prévoir les éléments pouvant être bloquant. Par exemple, dans le cas de collecte de données utilisateurs afin de réaliser un traitement, il convient de prévoir un délai pour effectuer les démarches auprès de la \gls{cnil}, prévoir un délai de validation du contrat avec les différents prestataires, les délais de réponse au sein de l'organisation afin de valider le projet...

Il convient aussi de définir les fonctionnalités par priorités. En effet, selon les délais impartis, il ne sera peut-être pas possible de voir toutes les fonctionnalités présentes dans la première version. Cela impliquerait un \gls{timetomarket} plus élevé, et donc des risques de voir d'autres concurrents se positionner sur le marché. Il est donc plus prudent de définir les fonctionnalités clés de l'application, et de prévoir les autres fonctionnalités dans les évolutions futures. On peut prioriser les fonctionnalités d'une application en plusieurs catégories.

\begin{itemize}
	\item Les \emph{must-have}. Ce sont les fonctionnalités sans lesquelles l'application n'est pas utilisable. Par exemple, si le service propose des achats, mais que le tunnel d'achat ne gère pas les paiements, il sera impossible de compléter la moindre commande, et par conséquent, le site ne sera pas utilisable.
	\item Les \emph{should have}
	\item Les \emph{could have}
	\item Les \emph{won't have}
\end{itemize}

%% https://fr.wikipedia.org/wiki/M%C3%A9thode_MoSCoW

\todo{Finir la partie planification}

- Définition d'un budget

- Négociation commerciale

- Définition risques juridiques / légaux (GDPR, CNIL...).

- Diagramme de Gant pour organisation projet, avec les grandes deadlines.

- Expression du besoin, rédaction cahier des charges. Objectif : supprimer ambiguité

Cahier des charges :permet de définir charte graphique \& architecture projet. Définit objectif. Poser au clair les besoins et spécificités des demandes. Priorisation. Définit qui a quel rôle? Contient des garanties pour le client (SLA...) \& l'entreprise (fonctionnalités hors scope \& co)

- Définition des priorités : Must, Should, May en vue de réaliser un MVP

Dans le cas d'une méthodologie \gls{cycle_v}, une seule phase de réalisation est faite, avant d'effectuer la validation avec le client. Cette méthodologie était beaucoup utilisée dans les années \todo{Chercher les années ou était utilisé Cycle en V}, mais avant l'inconvénient de provoquer une différence entre ce que le client souhaitait réellement et ce qu'il obtenait à la fin, plus ou moins conforme au cahier des charges. 

\newImage{0.5}{gestionProjet.jpg}{Gestion de projet, allégorie. Source : https://www.anyideas.net}{gestionProjet} 

Comme le montre l'image \ref{fig:gestionProjet}, souvent utilisée pour illustrer de manière humoristique les potentielles dérives de la gestion de projet, plus le nombre de personnes est important sur un projet et moins la communication est présente, plus les différences d'interprétation peuvent être importante. 

Une deuxième méthodologie de travail a vu le jour dans les années \todo{Chercher date début agilité}, et a pour objectif de combler ce manque de communication, et de faire des retours rapide au client. Dans cette méthodologie, la phase de réalisation est caractérisée sous forme de sprint, qui inclus non pas toutes les fonctionnalités demandées, mais seulement une partie, qui peut-être réalisable dans un temps convenu avec le client. Le but est de développer fonctionnalités par fonctionnalités, et de pouvoir présenter quelques choses de concret au client lors de chaque fin de sprint. Ainsi, à chaque itération, on se rapproche un peu plus de la version finale du produit, et on apporte constamment de la valeur ajoutée au produit.

\todo{Planification}

\subsection{Réalisation}

La réalisation de l'application intervient après la validation du besoin avec le client. Démarre alors la conception de la charte graphique, la réflexion sur les différentes fonctionnalités demandées, ainsi que les réalisations à proprement parler pour concevoir le site. La réalisation peut se dérouler de deux façons différentes.

- Réalisation graphique

- Plusieurs itérations

- Retour client, demande de précisions sur les fonctionnalités

- Développement de tests

\todo{Réalisation}

\subsection{Mise en recette \& Qualification}

- Déploiement en interne de l'application

- Permet de découvrir les premiers bug avec un environnement de production

- Retour sur ce qui était pensé par le client et ce qu'il veut au final (ce bouton, il le veut vraiment la le client en fait ? Ca fait pas très pratique ici..)

- Environnement pré-production / intégration \& co avant mise en production.

\todo{Mise en recette}

- Qualification : le client vérifie la conformité de l’application développée aux spécifications établies.

- Premier contact de l'application avec le client

- Evolution potentiellement lourde ("oh, au final je voulais plutôt ça, ça se fait bien ? Alors qu'il faut 10j pour le mettre en place")

- Potentiellement plusieurs déploiement avant validation finale

\subsection{Mise en production}

- Choix de la date important (ex: Transport, pas 10j avant la rentrée et tous les parents qui renouvellent l'abonnement de leur enfant.)

- Prévoir les imprévus : validation application mobile, propagation dns...

- Formations utilisateurs

- Promotion client du produit.

- Premier déploiement souvent "stressant"

- Besoin de métrique pour pouvoir déterminer le succès de l'application

\todo{Mise en production}

\subsection{Maintenance}

- Correction de bug au fur et à mesure

- Besoin de métrique pour pouvoir déterminer le statut de l'application, ses performances \& eventuels défauts.

- Suivi des logs 

- Système de remontée d'erreur organisée

\todo{Maintenance}

\subsection{Évolution}

Un projet qui n'évolue pas est un projet mort. Ainsi, il est nécessaire pour une application de se remettre continuellement en question, de chercher et d'implémenter de nouvelles idées, de faciliter la vie de l'utilisateur, de proposer de nouvelles fonctionnalités, de se mettre à jour vis à vis des dernières évolutions technologiques.

\todo{Evolution}

- Vie du projet

- Audit de sécurité, pour se tenir à jour

- Audit de performance

\subsection{Fin de vie d'un projet / Décommissionnement}

Éventuellement, un projet attendra sa fin de vie. Cela peut provenir de plusieurs raisons.

\begin{itemize}
	\item Un nouveau concurrent a pris des parts de marchés, entrainant une baisse de l'utilisation et de la rentabilité du service.
	\item Suite à une erreur de communication, ou une faille de sécurité, le projet est déserté par ses utilisateurs.
	\item Le projet est repris par un autre prestataire pour une refonte, ou tout simplement par choix du client (choix économique, mauvaise entente avec le prestataire actuel...).
\end{itemize}

Quoiqu'il en soit, il faut pouvoir gérer la fin de vie d'un projet correctement afin de repartir sur d'autres projets sereinement. 

Si l'application est reprise par un autre prestataire, il va se passer une période dite de réversibilité. Durant cette période, le prestataire actuel sera alors en communication afin de fournir les différents livrables pour assurer une bonne transition. 

Si l'application est en fin de vie, il faudra alors communiquer avec les utilisateurs, pour leur proposer de migrer vers des solutions alternatives, et leur proposer un moyen de récupérer leur données.

\subsection{Récapitulatif}

Dans cette section, nous avons vu les différentes étapes d'un projet, de la naissance du besoin, jusqu'à sa mise en production, en passant par les phases de réalisation et de dé-commissionnement. Ces étapes peuvent ainsi être résumée par le schéma ci-dessous.

\missingfigure{Schéma global reprenant les différents étapes du cycle de vie d'une application}

On voit alors que le cycle de vie d'un projet, au travers de plusieurs itérations permet d'améliorer constamment une application et de fournir des livrables de qualités au client.