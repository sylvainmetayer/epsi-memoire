\todo{Benefices}

\begin{itemize}
	\item KPI à trouver
	\item Confiance dans la livraison
	\item ROI avec rapidité de livraison
	\item Décharge l'équipe
	\subitem Plus de temps pour de la valeur métier
	\item L'automatisation peut permettre de relancer rapidement une activité défaillante (SLA / PRA / PCA)
\end{itemize}

\begin{itemize}
	\item Erreur développeur
	\item Fiabilité
	\item Preuve de qualité
\end{itemize}


Comment faire en sorte que ça marche dans la durée ? Des controles / supervisons périodique afin de checker que tout va bien.


Workflow : savoir ce qui est automatisé, comment c'est mis en oeuvre ,documentation des outils, ...
Faire en sorte que l'automatisation ne casse pas et que l'on en tire quelques chose, que l'on soit nouvel arrivant sur le projet, ou développeur déjà présent sur le projet.



Scalabilité : Docker / provisionner de nouveaux serveurs rapidement avec Ansible par exemple.

Déploiement : chaine de déploiement (dev/test/inte/preprod/prod) avec chacune ses spécificités

Exemple :

en dev, on souhaite des logs direct dans la console, en prod on les mets dans un fichier de log.
en dev, on veut le mode debug, en prod on le désactive.
