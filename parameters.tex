% Définitions des titres
\renewcommand\listoflistingscaption{Liste des codes sources}
\renewcommand\listingscaption{Code}

% pour switch auto papier / pdf
\newtoggle{paper}

% Pick one and comment the other one.
% \toggletrue{paper}
\togglefalse{paper}

\iftoggle{paper}{%
	% paper
	\usepackage[left=3cm,right=2cm,top=2cm,bottom=2cm]{geometry} % Pour le rapport imprimé 
}{%
	% electronic
	\usepackage[left=3cm,right=3cm,top=3cm,bottom=3cm]{geometry}
}

% Gestion des marges. Lien utile.
% https://www.debian-fr.org/t/latex-definir-un-paragraphe-en-dehors-des-marges/13991/3

% Réglages d'affichage
\setcounter{tocdepth}{5} % granularité de la table des matières
\setlength{\parindent}{2em} % On modifie l'indentation des paragraphes
\setlength{\parskip}{1em} % On modifie l'espace entre les paragraphes

%\widowpenalty=10000 % Pour les veuves et les orphelines
%\clubpenalty=10000 % Pour les veuves et les orphelines

\newcommand{\numerotationType}{arabic} %Type de numérotation : Roman ou arabic

% font settings
\setsansfont{CMU Sans Serif}%{Arial}
\setmainfont{CMU Serif}%{Times New Roman}
\setmonofont{CMU Typewriter Text}%{Consolas}
\defaultfontfeatures{Ligatures={TeX}}

%% FROM http://forum.mathematex.net/latex-f6/glossaire-t8654.html#p85429
\renewcommand{\glstextformat}[1]{#1*} % Affichage dans le corps du texte d'une entrée du glossaire
%\defglsdisplayfirst{#1#4\protect\footnote{#2}}

\defglsentryfmt{%
	\ifglsused{\glslabel}{%
		\glsgenentryfmt%
	}{%
		\glsgenentryfmt\protect\footnote{\glsentrydesc{\glslabel}}
	}%
}


\newtoggle{todoExplain}
\makeatletter % changes the catcode of @ to 11

\@ifpackagewith{todonotes}{disable}{
	\togglefalse{todoExplain}
}{
	\setlength{\marginparwidth}{2cm}
	\toggletrue{todoExplain}
}

\hypersetup{
	colorlinks=true, %colorise les liens
	breaklinks=true, %permet le retour à la ligne dans les liens trop longs
	urlcolor= blue, %couleur des hyperliens
	linkcolor= black, %couleur des liens internes
	citecolor=blue,    %couleur des liens de citations
	bookmarksopen=true,
	pdftoolbar=true,
	pdfmenubar=true,
	unicode=true,
	pdftitle={\@title},
	pdfauthor={\@author},
	pdfkeywords={epsi, memoire, automatisation},
	pdfsubject={Comment l'automatisation peut-elle permettre d'améliorer le cycle de vie d'une application ?},
	pdfcreator={\LaTeX}
}
\makeatother % changes the catcode of @ back to 12

% Pour les citations avec epigraphs
% \epigraphsize{\small}% Default
\setlength\epigraphwidth{.8\textwidth}
% \setlength\epigraphwidth{\textwidth}
\setlength\epigraphrule{1pt}

\setlength{\headheight}{14pt}
\presetkeys{todonotes}{inline}{}
