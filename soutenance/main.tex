% !TeX program = lualatex
% !TeX encoding = UTF-8
% !TeX spellcheck = fr_FR
% !TeX root = main.tex
% Source : https://fr.overleaf.com/latex/templates/template-defesa-ipb-utfpr/dcyxxfxxwfmv

\documentclass[xcolor={dvipsnames}]{beamer}

\usepackage[colorinlistoftodos,french]{todonotes}
\usepackage[french]{babel}
\usepackage{multimedia}

\setcounter{tocdepth}{1}

% Mettre les todos en lignes plutôt qu'en marge de la page
\presetkeys{todonotes}{inline}{}

\setbeamerfont{author}{size=\Large}

\mode<presentation> {

    \usetheme{Madrid}
    \usecolortheme[named=Farbe]{} % use a pre defined theme
    %\usecolortheme{ipbutfpr} % use a custom color theme
    % http://mcclinews.free.fr/latex/introbeamer/elements_diaporama.html

    \setbeamertemplate{caption}[numbered]
    \setbeamertemplate{frametitle continuation}{\gdef\beamer@frametitle{}}

    \setbeamertemplate{footline}{
        \leavevmode%
        \hbox{%
            %\begin{beamercolorbox}[wd=.4\paperwidth,ht=2.25ex,dp=1ex,center]{author in head/foot}%
            %    \usebeamerfont{author in head/foot}\insertshortauthor
            %\end{beamercolorbox}%
            \begin{beamercolorbox}[wd=1\paperwidth,ht=2.25ex,dp=1ex,center]{title in head/foot}%
                \usebeamerfont{title in head/foot}\insertshorttitle\hspace*{6em}
                \insertframenumber{} / \inserttotalframenumber\hspace*{1ex}
            \end{beamercolorbox}
        }%
        \vskip0pt%
    }

    \setbeamertemplate{navigation symbols}{}
    \usepackage{pgfpages}
    \setbeameroption{show notes on second screen=right}
	%\setbeameroption{show only notes}
    %\setbeameroption{hide notes}

	\definecolor{inversevideodessous}{rgb}{0.0, 0.28, 0.67}
	\definecolor{inversevideodessus}{rgb}{1,1,1}
	\usesectionheadtemplate
	{\colorbox{inversevideodessous}{\color{inversevideodessus} \insertsectionhead}}
	{\color{inversevideodessous} \insertsectionhead}
	
	\defbeamertemplate*{headline}{}
	{%
		\begin{beamercolorbox}[ht=1.875ex,dp=0.75ex]{section in head/foot}%
			\insertsectionnavigationhorizontal{\paperwidth}{}{}%
		\end{beamercolorbox}%
		\begin{beamercolorbox}[ht=1.875ex,dp=0.75ex]{subsection in head/foot}
			\usebeamerfont{subsection in head/foot}
			\insertsubsectionnavigationhorizontal{\paperwidth}{}{\hskip0pt plus1filll}%
		\end{beamercolorbox}%
	}    
}

\title[Automatisation \& amélioration du cycle de vie d'une application]{Comment l’automatisation peut-elle permettre d’améliorer le cycle de vie d’une application ?}
\author{Sylvain METAYER}

\institute{\normalsize
    \begin{figure}[htb]
        \centering
        \includegraphics[width=0.35\textwidth]{img/onepoint.jpg}
        \includegraphics[width=0.35\textwidth]{img/epsi.jpg}
    \end{figure}
}

\date{\small 4 Septembre 2019 \\ 11h}

\begin{document}
    \frame[plain]{\titlepage}
    
    \begin{frame}[plain]{Plan}
    	\tableofcontents
	\end{frame}      

	%-----SLIDES-----%
	\todo{Introduction, 5-6 pages}

\todo{Quand la partie introduction sera terminée, supprimer les parties.}

\subsection{Accroche}
L'automatisation a toujours été perçue comme un moyen de gagner en productivité, temps, et donc de rendre des projets toujours plus rentable.
	
Les projets informatiques sont de plus en plus nombreux, que cela soit des logiciels de bureau, des applications web, ou encore avec les nouveaux terminaux, des applications mobile, tablettes ou même pour montres connectées.
	
Les projets augmentent donc en quantité, mais augmentent-ils en qualité ? Leur fiabilité n'est en effet pas toujours optimale. 
	
Combien de projet sont encore déployé manuellement car aucune automatisation n'est présente sur le projet ? 
	
En plus d'une perte de temps, parfois importante, cela engendre un stress au niveau des équipes de développeurs, qui à chaque livraison redoute les régressions qui pourraient survenir ou encore les bugs de déploiement.

L'automatisation peut également permettre d'améliorer l'arrivée d'un nouveau développeur sur un projet. Il n'est en effet par rare de voir des projets ou la configuration de l'environnement requiert à elle seule plusieurs jours, sans que le développeur puisse vraiment commencer à travailler.
	
L'automatisation va permettre d'améliorer la fiabilité ainsi que la confiance des développeurs et clients dans le projet, puisque des tests automatisés ainsi qu'une chaine d'industrialisation complètement automatisée permet ainsi de déployer avec confiance une application.
	
Nous allons donc tenter de répondre à la problématique suivante :
	
{\LARGE \problematique}

\subsection{Définition}

Avant de continuer, il convient de s'attarder sur ce qu'est l'automatisation.

Selon le Larousse, l'automatisation est le 

\begin{quote}
fait d'automatiser l'exécution d'une tâche, d'une suite d'opérations...
\end{quote}

\subsection{Historique}

Idées : 
	
- Mode opératoire suivi religieusement, 

- script expect

- Comment déployait-on avant ?

-  ...


\textit{L’introduction doit remplir une fonction traditionnelle : délimiter et présenter le projet ou la mission et son contexte professionnel, annoncer les parties principales du développement. L’introduction représente environ un dixième du mémoire. Il faut absolument insister sur la bonne impression qu’elle doit donner au lecteur comme premier et décisif contact avec le mémoire.}

\subsection{Que peut-on automatiser ?}

\subsection{Pourquoi automatiser ?}

Quels en sont les avantages ?

- éviter erreurs développeurs

- éviter script executé avec mauvais paramètres

\subsection{Présentation entreprise \& mission}

L'entreprise dans laquelle j'effectue mon alternance depuis septembre 2017 est \onepoint. 

\xmakefirstuc{\onepoint{}} est une \gls{esn} à taille humaine. Son domaine d'activité est d'accompagner ses clients dans leur transformation numérique, 

C'est une \gls{SAS} disposant de 14 implantations dans le monde. L'entreprise a effectué en 2018 un chiffre d'affaires de 300 million d'euros.

Elle dispose de 2300 collaborateurs, en moyenne agé de 33 ans.

\xmakefirstuc{\onepoint{}} se compose de plusieurs communautés.

\begin{itemize}
	\item Des communautés \emph{régions}, permettant de regrouper les collaborateurs par leur proximité géographique.
	\item Des communautés \emph{expertise}, regroupant l'expertise de chacun, et permettant à tous de progresser. On y retrouve par exemple la communauté Sécurité ou encore Architecture.
	\item Des communautés \emph{support}, tel que la \gls{DSI}, ou les Ressources Humaines, nécessaire au fonctionnement de l'entreprise.
	\item Des communautés \emph{métiers}, regroupant des personnes maitrisant les aspects métiers des différents clients, ainsi que les contraintes de ces métiers. Cela peut par exemple être les métiers des Assurances, des Banques, des Télécoms... 
\end{itemize}

Ainsi, lors du développement d'un projet, toutes ces communautés sont utilisés, afin de tirer le meilleur d'entre elle et de regrouper les personnes les plus aptes à réaliser le projet.

Cela implique aussi que chaque collaborateur peut ainsi appartenir à une ou plusieurs communautés, selon ses compétences, expérience et localisation.

De plus, \onepoint{} se définit autour de 5 valeurs.

\begin{itemize}
	\item L'authenticité
	\item L'élégance
	\item L'ouverture, que cela soit au niveau des clients, de tout horizon, ou en interne, en faisant en sorte que les différentes communautés soit sans frontière, pour permettre à chacun d'évoluer
	\item L'engagement, afin de garantir la qualité des différents projets réalisés.
	\item Le courage, pour oser sortir des sentiers battus, et continuellement proposer de nouvelles solutions innovantes.
\end{itemize}

\subsubsection{Historique} 

\xmakefirstuc{\onepoint{}} a été créé en 2002, par David Layani.

De 2003 jusqu'en 2007, elle va s'ouvrir à l'international, avec l'ouverture de bureaux au Canada, en Chine et en Tunisie.

En 2008, elle étend sa position en France, avec l'ouverture de deux centres de production, à Bordeaux et Nantes.

En 2015, \onepoint{} continue son développement international au Luxembourg, en Belgique et en Hollande, avant de racheter VisionIT Group.

En 2018, \onepoint{} ouvre des bureaux à Lyon et en Australie, et rachète également Weave ainsi que Géronimo, acteur important dans la conception d'application mobile.

\subsubsection{Réalisations}

\todo{Ajouter des réalisations de onepoint}

\subsubsection{Contexte de l'alternance}

Projet Nouvelle Aquitaine, chaine d'industrialisation pour pouvoir permettre déploiement de multiple sites Drupal.

\textit{Projet \bv{}, ou il y a une architecture actuelle qui n'est pas satisfaisante pour X raisons (reproductibilité...), et qui doit être changé}.

\textit{Elaborer un schéma directeur à partir d’orientations stratégiques} => Conduite de changement


\subsection{Annonce du plan}
	\section[Cycle de vie]{Cycle de vie d'une application}
\subsection{Définition}
\begin{frame}{\subsecname}
	\begin{block}{}
		Ensemble d'étapes intervenant au cours du développement d'un projet informatique
	\end{block}
\end{frame}

\subsection{Cas de la Nouvelle-Aquitaine}
\begin{frame}{Le projet}
	\begin{block}{Drupal - Des portails web}
		\note{Pour décrire les différents services et aides offerte par la région. Chacun s'adresse à un public différent.\\ Certains sont une refonte, d'autres non.\\ Pourquoi Drupal ? Car NAQ déjà formée niveau rédacteur sur ce CMS et souhaite garder le même.}
		\begin{itemize}
			\item Transport
			\item Guide des aides
			\item Jeunes
			\item Régie d'Information
			\item Entreprise
			\item ...
		\end{itemize}
	\end{block}
\end{frame}

\begin{frame}{Cycle de vie des portails de la Nouvelle-Aquitaine}
	\begin{center}
		\includegraphics[width=0.80\textwidth]{img/cycle-vie-naq.png}
	\end{center}
	\note{Chaque portail à son propre cycle de vie, de définition besoin à prod en passant par bug fix \\}
	\note{Important : parler problème. Un problème peut être une évolution qui n'a pas / mal été chiffré avant. Cela peut être un incident (données mal saisie entrainant erreur, ou encore coupure du serveur) ou cela peut-être un bug. Chacun est traité et chiffré différemment.}
\end{frame}
	\section[Automatisation]{Automatisation du cycle de vie d'une application}

\subsection{Définition}
\begin{frame}{\subsecname}
	\begin{block}{Selon le Larousse}
		\frquote{Exécution totale ou partielle de tâches techniques par des machines fonctionnant sans intervention humaine.}
	\end{block}
\end{frame}

\subsection{Pourquoi ?}
\begin{frame}{\subsecname}
	\begin{overprint}
		\onslide<1> 
		\begin{block}{Besoin de factorisation -- harmonisation}
			\note[item]{Emergence de besoins communs selon les portails, mais pas de réelles solutions pour mutualiser les solutions. Copier coller entre les portails qui faisait que les mises à jour étaient compliquées. Besoin d'harmonisation entre les portails}
			\missingfigure{Illustration}
		\end{block}
		\onslide<2> 
		\begin{block}{Mise à disposition lente \& peu fréquente}
			\note[item]{Une ou deux personnes disponible pour déployer. Déploiement manuel pouvant entrainer des erreurs.}
			\missingfigure{Illustration}
		\end{block}
		\onslide<3> 
		\begin{block}{Pas de réel environnement de test}
			\note[item]{Staging qui était assez instable, déployé par qui pouvait selon les moments/compétences. Fait que les phases de tests étaient parfois sur les postes des développeurs au lieu d'être sur une vraie recette interne avant la recette client.}
			\missingfigure{Illustration}
		\end{block}
	\end{overprint}
\end{frame}

\subsection{Gestion de projet}
\begin{frame}{KPI}
	\begin{columns}[onlytextwidth]
		\column{.4\textwidth}
		\begin{block}{}
			\todo{trouver d'autres exemple de KPI NAQ en réponses à d'éventuelles questions.}
			\begin{itemize}
				\item Temps de déploiement réduit (30min -- 1h $\rightarrow$ 5min)
				\note[item]{Avant : construction package de livraison à la main, upload, potentielle erreur de droits... Du temps gagné, donc de l'argent }
				\pause
				\item Livrer de la valeur métier plus rapidement
				\note[item]{Réduire délai entre poste développeur -- test -- recette -- preprod -- prod et passer plus de temps sur de la valeur métier que de la livraison}
			\end{itemize}
		\end{block}
		\column{.5\textwidth}
		\includegraphics[width=0.8\linewidth]{img/devops-objective.jpg}
	\end{columns}
\end{frame}

\begin{frame}{Work In Progress}
	\begin{block}{Objectif : le réduire}
	\end{block}
	\note{Exemple sur les projets NAQ, avant demande qui pouvait trainer plusieurs semaines en attente. Maintenant, on a des demandes qui ne sont pas travaillées tant qu'elles ne sont pas clairement définies par le client.\\ JIRA pour s'organiser dans les tâches. Permet de voir que si quelqu'un sur plusieurs sujet, il vaut mieux le laisser se concentrer sur un et dispatcher les autres à d'autres personnes ou remettre ces tâches à une date ultérieure \\ Automatisation permet de réduire les bouchons d'étranglement de déploiement de tous les portails en même temps pour tester.}
	\centering \movie[width=8cm,height=5cm]{}{img/wip.gif}
\end{frame}

\subsection[Développement]{Pratiques de développement}
\begin{frame}{Code review}
	\centering \includegraphics[width=0.50\textwidth]{img/code-review.png}
	\note[item]{Entraine une plus grande communication au sein de l'équipe, et une résilience de l'information même s'il peut y avoir des différents comme l'illustre l'image ci-contre}
\end{frame}

\begin{frame}{Environnement local de développement}
	\begin{overprint}
		\onslide<1> 
		\begin{block}{Installation d'un portail localement -- Avant}
			\note[item]{README trop long, pas maintenu, pas assez détaillé, qui se contredise, différentes selon les portails}
			\begin{columns}[onlytextwidth]
				\column{.4\textwidth}
				\begin{itemize}
					\item Chaque portail à ses spécificités
					\item \textasciitilde 15 étapes
					\item 20 minutes $\longleftrightarrow$ 2 heures
				\end{itemize}
				\column{.5\textwidth}
				\movie[width=6cm,height=6cm]{}{img/before-2.gif}
			\end{columns}
		\end{block}
		\onslide<2> 
		\begin{block}{Installation d'un portail localement -- Maintenant}
			\note[item]{Même procédure d'installation pour tous les portails. Passage de Xh à 10min, temps de téléchargement des dépendances compris. Pile le temps de prendre un café !}
			\begin{columns}[onlytextwidth]
				\column{.4\textwidth}
				\begin{itemize}
					\item Identique pour tous les portails
					\item 5 commandes
					\item 10 minutes
				\end{itemize}
				\column{.5\textwidth}
				\movie[width=6cm,height=6cm]{}{img/after3.gif}
			\end{columns}
		\end{block}
		\onslide<3> 
		\todo{Soit étoffer cette slide, soit la supprimer}
		\begin{block}{Bonnes pratiques}
			\begin{itemize}
				\item PSR PHP (Propose a Standards Recommendation for PHP)
			\end{itemize}
		\end{block}
	\end{overprint}
\end{frame}

\subsection{Tests}
\begin{frame}{\subsecname}
	\begin{block}{Une mise en place timide}
		\begin{itemize}
			\item POC tests Drupal
			\item Écriture de \frquote{modèle de tests} pour encourager la pratique
			\item Build qualité -- test de mise à jour
		\end{itemize}
	\end{block}
\end{frame}

\subsection{Conduite de changement}
\begin{frame}{\subsecname}
	\note[item]{Changement petit à petit, pour permettre une migration en douceur.}
	\begin{columns}[onlytextwidth]
		\column{.4\textwidth}
		\begin{itemize}
			\item Phase de réflexion commune
			\note[item]{validation par plusieurs architectes applicatifs}
			\item Accompagnement
			\note[item]{pour les migrations/ changement d'habitude}
			\item Dépréciation ancienne structure de code sur 2 mois
			\note[item]{garantir que il n'y aura pas de perte de code ni d'historique lors de la transition}
		\end{itemize}
		\column{.6\textwidth}
		\begin{figure}
			\includegraphics[width=0.6\linewidth]{img/rome.jpg}
			\captionsetup{labelformat=empty}
			\caption{https://twitter.com/drzarrow/status/479245604552704000}
			\label{fig:rome}
		\end{figure}
	\end{columns}
	%Cela ne s'est pas fait en un jour. Changement petit à petit, pour inclure de plus en plus de bonne pratiques \& et d'automatisation
\end{frame}

	\section{Mise en place CI}
\subsection[Architecture]{Architecture projet}
\begin{frame}{\subsecname}
	\begin{overprint}
		\onslide<1> 
		\begin{block}{Avant}
			\note<1>[item]{Chaque portail différent. Maintenance difficile, impossible de réutiliser des modules d'un site dans un autres.}
			\centering \includegraphics[width=0.25\textwidth]{img/before-drupal.png}
		\end{block}
		\onslide<2> 
		\begin{block}{Après}
			\note<2>[item]{Mise en place d'un modèle de référence pour tous les portails. Regroupement modules communs dans un dépôt à part qui sera exposé sur un artifactory pour être réutilisé par les autres sites, en gérant les versions. Travail de migration de tous les portails existants}
			\centering \includegraphics[width=0.50\textwidth]{img/after-drupal.png}
		\end{block}
	\end{overprint}
\end{frame}

\subsection{Versions}
\begin{frame}{Gestion de versions}
	\note[item]{Déploiement auto = version auto. Besoin de savoir ce qui est déployé à un instant T car plusieurs déploiement peuvent être fait par jour.}
	\note[item]{avant: tag git manuel, mais pouvait être oublié sur l'un des trois dépôts. Inconvénient : si erreur de livraison, on ne pense pas forcément à retaguer les trois dépôt après le correctif. Peut être source de problème.}
	\note[item]{Maintenant, modules commun mis à jour à chaque push des développeurs. On peut donc savoir la version d'un site et des modules communs qu'il utilisent à tout instant.}

	\begin{columns}[onlytextwidth]
		\column{.4\textwidth}
		\centering \includegraphics[width=1\textwidth]{img/final-version.png}
		\pause
		\column{.5\textwidth}
		\begin{block}{}
			\begin{itemize}
				\item v1.0.0.1
				\item v1.0.0.2
				\item v1.0.1.0
				\item ...
			\end{itemize}
		\end{block}
	\end{columns}
\end{frame}

\subsection{Jenkins}
\begin{frame}{\subsecname}
	\note<1>[item]{Organisé par portail, job identiques, versionnés pour maintenabilité}
	\note<2>[item]{Différents type de job : après push développeur, rapide, chaque nuit, qualité, déploiement, à la demande, possible par toute l'équipe.}
	\note<3>[item]{log par étape afin de savoir les étapes réussies ou non}
	\note<3>[item]{Alerte sur Teams (outil com interne) sur les résultats des builds}
	\begin{overprint}
		\onslide<1>
		\begin{block}{Par projet}
			\centering \includegraphics[width=0.8\textwidth]{img/job-naq.png}
		\end{block}
		\onslide<2>
		\begin{block}{Différents besoins}
			\centering \includegraphics[width=0.8\textwidth]{img/job-starter.png}
		\end{block}
		\onslide<3>
		\begin{block}{Suivi des étapes}
			\centering \includegraphics[width=0.8\textwidth]{img/job-regie-staging.png}
		\end{block}
	\end{overprint}
\end{frame}

\subsection{Déploiement}
\begin{frame}{\subsecname}
	\note[item]{Build package de livraison avec ses dépendances, pour éviter erreurs futures \& comportement non souhaités (coupure réseau...) au plus tôt}
	\note[item]{Utilisation d'ansible pour décrire les tâches exécutées. Egalement versionné}	\centering \includegraphics[width=1\textwidth]{img/ansible-recap.png}
\end{frame}

\subsection{Sauvegarde}
\begin{frame}{\subsecname}
	\note[item]{Activable par option, permet sauvegarder base données + fichiers. Restauration pas encore automatisée mais prévue.}
\end{frame}

\subsection{Sécurité}
\begin{frame}{Sécurité - Bastion}
	\note[item]{Clé SSH, compte nominatif, permissions restreinte, user de déploiement}
	\note[item]{bastion ssh qui permet de se couper quand pas nécessaire}
	\pause
	\centering \includegraphics[width=0.60\textwidth]{img/bastion.png}
\end{frame}

	C'est maintenant la fin de ce document. Les différentes étapes du cycle de vie d'une application ont été détaillée et nous avons vu qu'il n'existait pas d'automatisation \frquote{miracle} permettant de répondre à tous les besoins mais qu'il s'agissait d'un assemblage d'outils, chacun effectuant une tâche dédiée. De plus, la diversité des projets a montré qu'il est impossible de réaliser \frquote{un outil pour les gouverner tous}.

Nous avons aussi vu que les besoins d'automatisation était de plus en plus utilisés, du fait que la complexité des applications grandit de jour en jour\footnote{bien que cela ne soit pas toujours justifié}. L'objectif est de fournir la meilleure expérience utilisateur à l'utilisateur final, tout en permettant des livraisons fréquentes et rapide afin d'apporter de la valeur ajoutée au produit.


Nous allons donc reprendre dans cette conclusion les différents avantages à mettre en place une démarche d'automatisation sur un projet.

Tout d'abord, le but est de réduire les erreurs humaines lors des phases de développement tout en permettant d'automatiser les tâches répétitives. La réduction des erreurs entraine alors une meilleure fiabilité de l'application, via la mise en place de tests automatisés. Cette confiance dans l'application permet alors de livrer des fonctionnalités plus rapidement et contribue à réduire le \gls{timetomarket} pour offrir de nouvelles fonctionnalités aux utilisateurs, ou corriger les bugs plus rapidement.

De plus, l'automatisation va également permettre une reprise d'activité plus rapide et peut donc être justifiée dans les \gls{PRA} et les \gls{PCA}, en insistant sur le fait qu'une relance rapide de l'activité peut permettre de limiter les pertes financières. Cela en dù au fait que l'automatisation permet le provisionnement de serveur de façon automatique, garantissant la reproductibilité des étapes à réaliser pour déployer une application, avec des outils tel qu'Ansible.

\todo[color=yellow]{Conclusion, 7 pages}

\subsection*{Bénéfices}

\todo[color=orange]{Benefices}

moins d'erreurs développeur, plus de fiabilité, time to market réduit, ROI avec confiance et livraison plus rapide. Relancer activité défaillante (SLA/PCA/PRA). Plus de temps pour la valeur métier. 

Documentation : arrivée sur un projet, démarrage rapide.

Scalabilité : docker - provisionnemnt rapide de novueaux serveurs avec ansible.

en dev, on souhaite des logs direct dans la console, en prod on les mets dans un fichier de log.
en dev, on veut le mode debug, en prod on le désactive.

Cela peut être perçu comme une perte de temps par certaines personnes (hiérarchie, manager...), qui y verrons la une perte d'argent et de temps par exemple. Il faut alors pouvoir montrer que cela est rentable, au travers de \gls{KPI} bien déterminés.

Le fait d'automatiser des processus permet de gagner du temps, et par conséquent de l'argent et de consacrer ses efforts à d'autres taches qui peuvent apporter de la valeur métier. Le coût horaire libéré, calculer à partir de combien de temps il est rentable.

Facteur humain : Le temps libéré par l'automatisation des tâches peut permettre de souder les liens d'une équipe et d'améliorer les relations de cette dernière. Cela libère du temps pour du team building par exemple.

Métrique - KPI - temps de déploiement / nombre d'incident / uptime / Nombre de build KO / Nombre de build OK ... permettent de définir l'impact des services mis en place sur le S.I

\begin{itemize}
	\item temps de déploiement
	\item taux de déploiement succès
	\item couverture de code
	\item tests au verts
	\item derniers build KO
\end{itemize}

\subsection*{Limites}

\todo[color=orange]{Limites}

Sur-qualité, sur-optimisation

Aucun intérêt si les tests ne sont pas fiable.

Dépend du client, de l'environnement, demande de la flexibilité, délai de mise en place, ROI

Pré-requis à l'automatisation

Attention, auto != vérité. récompenses / punition : il y a des phénommènes de triches / optimisation pour avoir le meilleur score.
% https://www.matuzo.at/blog/building-the-most-inaccessible-site-possible-with-a-perfect-lighthouse-score/

L'automatisation ne peut être utilisée, ou sera risquée, ou plus à même d'amener des régression si de mauvaise pratique sont présentes : variable codée en dur à la place de variable d'environnement

De plus, il faut une certaine organisation.

Parler de l'importance de l'implication du client. 

Ex :  Rédaction de SFD, qui évolue tout les 4 matins, et demande un changement dans l'architecture => automatisation perdante.

nécessite certains prérequis : 

- création d'environnemnet à la volée. Si on doit attendre des mois pour ça, ça n'a pas de sens. Solution => serverless avec à la demande ? 

\subsection*{Ce qui n'est pas encore automatisé}

\todo{ce qui n'est pas encore automatisé}

selon les entreprises, dépend du besoin.

\newImage[H]{0.7}{time-worth.png}{Temps pour automatiser tâches - \url{https://xkcd.com/1205/}}{time-worth}

\subsection*{Les possiblités d'automatisation dans le futur}

\todo{Parler de ce qui n'est pas encore automatisé, et les différentes pistes d'automatisation possible dans le futur.}

	%----------------%
	
    %----SAMPLES----%
    %\begin{frame}{slide autre}

du blabla

\end{frame}
    %\begin{frame}{slides}

\begin{figure}[htb]
	\centering
	\includegraphics[width=0.3\textwidth]{../img/devops-mvp.jpg}
	\caption{Image}
	\label{fig:onepoint}
\end{figure}

\end{frame}

    %\begin{frame}{Slide}

\begin{block}{un bloc}
    des trucs
\end{block}

\begin{block}{autre bloc}
	blablalba
\end{block}

\end{frame}

    %--------------%

\end{document}
