\section{Mise en place CI}
\subsection[Architecture]{Architecture projet}
\begin{frame}{\subsecname}
	\begin{overprint}
		\onslide<1> 
		\begin{block}{Avant}
			\note<1>[item]{Chaque portail différent. Maintenance difficile, impossible de réutiliser des modules d'un site dans un autres.}
			\centering \includegraphics[width=0.3\textwidth]{img/before-drupal.png}
		\end{block}
		\onslide<2> 
		\begin{block}{Après}
			\note<2>[item]{Mise en place d'un modèle de référence pour tous les portails. Regroupement modules communs dans un dépôt à part qui sera exposé sur un artifactory pour être réutilisé par les autres sites, en gérant les versions. Travail de migration de tous les portails existants}
			\centering \includegraphics[width=1\textwidth]{img/after-drupal.png}
		\end{block}
	\end{overprint}
\end{frame}

\subsection{Versions}
\begin{frame}{Gestion de versions}
	\note[item]{Déploiement auto = version auto. Besoin de savoir ce qui est déployé à un instant T car sinon on peut se retrouver sans savoir ce qui est déployé (sans pour autant être dans le cas de yoda).}
	\note[item]{avant: tag git manuel, mais pouvait être oublié sur l'un des trois dépôts. Peut être source de problème.}
	\note[item]{Maintenant, auto tag + version mise à jour à chaque déploiement}

	\begin{columns}[onlytextwidth]
		\column{.4\textwidth}
		\centering \includegraphics[width=1\textwidth]{img/version-yoda.jpg}
		\pause
		\column{.5\textwidth}
		\begin{block}{}
			\begin{itemize}
				\item v1.0.0.1
				\item v1.0.0.2
				\item v1.0.1.0
				\item ...
			\end{itemize}
		\end{block}
	\end{columns}
\end{frame}

\subsection{Intégration continue}
\begin{frame}{\subsecname}
	\note<2>[item]{Différents type de job : après push développeur, rapide, chaque nuit, qualité, déploiement, à la demande, possible par toute l'équipe.}
	\note<3>[item]{log par étape afin de savoir les étapes réussies ou non}
	\note<3>[item]{Alerte sur Teams (outil com interne) sur les résultats des builds}
	\begin{overprint}
		\onslide<1>
		\begin{block}{Jenkins}
			\centering \includegraphics[width=0.25\textwidth]{img/jenkins.png}
		\end{block}
		\onslide<2>
		\begin{block}{Différents besoins}
			\centering \includegraphics[width=1\textwidth]{img/job-transport.png}
		\end{block}
		\onslide<3>
		\begin{block}{Suivi des étapes}
			\centering \includegraphics[width=1\textwidth]{img/job-transport-staging.png}
		\end{block}
	\end{overprint}
\end{frame}

\subsection{Déploiement}
\begin{frame}{\subsecname}
	\note[item]{Build package de livraison avec ses dépendances, pour éviter erreurs futures \& comportement non souhaités (coupure réseau...) au plus tôt}
	\note[item]{Utilisation d'ansible pour décrire les tâches exécutées. Egalement versionné}	\centering \includegraphics[width=1\textwidth]{img/ansible-recap.png}
\end{frame}

\subsection{Sauvegarde}
\begin{frame}{\subsecname}
	\note[item]{Activable par option, permet sauvegarder base données + fichiers. Restauration pas encore automatisée mais prévue.}
	\begin{figure}
		\includegraphics[width=0.6\linewidth]{img/backup.jpg}
		\captionsetup{labelformat=empty}
		\caption{Photo by Markus Spiske on Unsplash}
		\label{fig:backup}
	\end{figure}
\end{frame}

\subsection{Sécurité}
\begin{frame}{Sécurité - Bastion}
	\note[item]{Clé SSH, compte nominatif, permissions restreinte, user de déploiement}
	\note[item]{bastion ssh qui permet de se couper quand pas nécessaire}
	\centering \includegraphics[width=0.60\textwidth]{img/bastion.png}
\end{frame}
