\section[Cycle de vie]{Cycle de vie d'une application}
\subsection{Définition}
\begin{frame}{\subsecname}
	\begin{block}{}
		Ensemble d'étapes intervenant au cours du développement d'un projet informatique
	\end{block}
\end{frame}

\subsection{Cas de la Nouvelle-Aquitaine}
\begin{frame}{Le projet}
	\begin{block}{Drupal - Des portails web}
		\note{Pour décrire les différents services et aides offerte par la région. Chacun s'adresse à un public différent.\\ Certains sont une refonte, d'autres non.\\ Pourquoi Drupal ? Car NAQ déjà formée niveau rédacteur sur ce CMS et souhaite garder le même.}
		\begin{itemize}
			\item Transport
			\item Guide des aides
			\item Jeunes
			\item Régie d'Information
			\item Entreprise
			\item ...
		\end{itemize}
	\end{block}
\end{frame}

\begin{frame}{Cycle de vie des portails de la Nouvelle-Aquitaine}
	\begin{center}
		\includegraphics[width=0.80\textwidth]{img/cycle-vie-naq.png}
	\end{center}
	\note{Chaque portail à son propre cycle de vie, de définition besoin à prod en passant par bug fix \\}
	\note{Important : parler problème. Un problème peut être une évolution qui n'a pas / mal été chiffré avant. Cela peut être un incident (données mal saisie entrainant erreur, ou encore coupure du serveur) ou cela peut-être un bug. Chacun est traité et chiffré différemment.}
\end{frame}